\documentclass{beamer}
%\DefineVerbatimEnvironment{Highlighting}{Verbatim}{commandchars=\\\{\}}
% Add ',fontsize=\small' for more characters per line
\usepackage{MnSymbol,wasysym}

\newenvironment{Shaded}{}{}
\newcommand{\KeywordTok}[1]{\textcolor[rgb]{0.00,0.44,0.13}{\textbf{{#1}}}}
\newcommand{\DataTypeTok}[1]{\textcolor[rgb]{0.56,0.13,0.00}{{#1}}}
\newcommand{\DecValTok}[1]{\textcolor[rgb]{0.25,0.63,0.44}{{#1}}}
\newcommand{\BaseNTok}[1]{\textcolor[rgb]{0.25,0.63,0.44}{{#1}}}
\newcommand{\FloatTok}[1]{\textcolor[rgb]{0.25,0.63,0.44}{{#1}}}
\newcommand{\CharTok}[1]{\textcolor[rgb]{0.25,0.44,0.63}{{#1}}}
\newcommand{\StringTok}[1]{\textcolor[rgb]{0.25,0.44,0.63}{{#1}}}
\newcommand{\CommentTok}[1]{\textcolor[rgb]{0.38,0.63,0.69}{\textit{{#1}}}}
\newcommand{\OtherTok}[1]{\textcolor[rgb]{0.00,0.44,0.13}{{#1}}}
\newcommand{\AlertTok}[1]{\textcolor[rgb]{1.00,0.00,0.00}{\textbf{{#1}}}}
\newcommand{\FunctionTok}[1]{\textcolor[rgb]{0.02,0.16,0.49}{{#1}}}
\newcommand{\RegionMarkerTok}[1]{{#1}}
\newcommand{\ErrorTok}[1]{\textcolor[rgb]{1.00,0.00,0.00}{\textbf{{#1}}}}
\newcommand{\NormalTok}[1]{{#1}}

%\usepackage[table]{xcolor}
\mode<presentation> {
  \usetheme{Boadilla}
%  \usetheme{Pittsburgh}
%\usefonttheme[2]{sans}
\renewcommand{\familydefault}{cmss}
%\usepackage{lmodern}
%\usepackage[T1]{fontenc}
%\usepackage{palatino}
%\usepackage{cmbright}
  \setbeamercovered{transparent}
\useinnertheme{rectangles}
}
%\usepackage{normalem}{ulem}
%\usepackage{colortbl, textcomp}
\setbeamercolor{normal text}{fg=black}
\setbeamercolor{structure}{fg= black}
\definecolor{trial}{cmyk}{1,0,0, 0}
\definecolor{trial2}{cmyk}{0.00,0,1, 0}
\definecolor{darkgreen}{rgb}{0,.4, 0.1}
\usepackage{ifxetex,ifluatex}

\usepackage{array}
\beamertemplatesolidbackgroundcolor{white}  \setbeamercolor{alerted
text}{fg=red}
\newtheorem{assumption}{Assumption}

\setbeamertemplate{caption}[numbered]\newcounter{mylastframe}

%\usepackage{color}
\usepackage{tikz}
\usetikzlibrary{arrows}
\usepackage{colortbl}
\usepackage{fancyvrb}
\usepackage{multirow}

%\usepackage[usenames, dvipsnames]{color}
%\setbeamertemplate{caption}[numbered]\newcounter{mylastframe}c
%\newcolumntype{Y}{\columncolor[cmyk]{0, 0, 1, 0}\raggedright}
%\newcolumntype{C}{\columncolor[cmyk]{1, 0, 0, 0}\raggedright}
%\newcolumntype{G}{\columncolor[rgb]{0, 1, 0}\raggedright}
%\newcolumntype{R}{\columncolor[rgb]{1, 0, 0}\raggedright}

%\begin{beamerboxesrounded}[upper=uppercol,lower=lowercol,shadow=true]{Block}
%$A = B$.
%\end{beamerboxesrounded}}
\renewcommand{\familydefault}{cmss}
%\usepackage[all]{xy}

\usepackage{tikz}
\usepackage{lipsum}


 \newenvironment{changemargin}[3]{%
 \begin{list}{}{%
 \setlength{\topsep}{0pt}%
 \setlength{\leftmargin}{#1}%
 \setlength{\rightmargin}{#2}%
 \setlength{\topmargin}{#3}%
 \setlength{\listparindent}{\parindent}%
 \setlength{\itemindent}{\parindent}%
 \setlength{\parsep}{\parskip}%
 }%
\item[]}{\end{list}}
\usetikzlibrary{arrows}
%\usepackage{palatino}
%\usepackage{eulervm}
\usecolortheme{lily}

\newtheorem{com}{Comment}
\newtheorem{lem} {Lemma}
\newtheorem{prop}{Proposition}
\newtheorem{thm}{Theorem}
\newtheorem{defn}{Definition}
\newtheorem{cor}{Corollary}
\newtheorem{obs}{Observation}
 \numberwithin{equation}{section}
\newtheorem{iass}{Identification Assumption}
\newtheorem{ires}{Identfication Result}
\newtheorem{estm}{Estimand}
\newtheorem{esti}{Estimator}
\newcommand{\indep}{{\bot\negthickspace\negthickspace\bot}}

%Box Types


\title[Causal Inference] % (optional, nur bei langen Titeln nötig)
{Causal Inference}

\author{Justin Grimmer}
\institute[University of Chicago]{Associate Professor\\Department of Political Science \\  University of Chicago}
\vspace{0.3in}

\date{April 16th, 2018}

\begin{document}
\begin{frame}
\titlepage
\end{frame}


%
%
%\section{Experiments vs Observational
%Studies}\label{experiments-vs-observational-studies}
%
%\begin{frame}{Experiments vs Observational Studies}
%
%\begin{figure}[ht] \centering
%    \includegraphics[width = .8 \linewidth]{images/cartoon.jpg}
%\end{figure}
%
%Lancet 2001: negative correlation between coronary heart disease mor-
%tality and level of vitamin C in bloodstream (controlling for age,
%gender, blood pressure, diabetes, and smoking)
%
%\end{frame}
%
%\begin{frame}{Experiments vs Observational Studies}
%
%\begin{figure}[ht] \centering
%    \includegraphics[width = .85 \linewidth]{images/cartoon.jpg}
%\end{figure}
%
%Lancet 2002: no effect of vitamin C on mortality in controlled placebo
%trial (controlling for nothing)
%
%\end{frame}

\frame{
%\frametitle{Experiments vs. Observational Studies}
\vspace{-.25cm}
\begin{figure}[ht] \centering
    \includegraphics[width = .85 \linewidth]{images/cartoon.jpg}
\end{figure}
\begin{overprint}
\onslide<1>
Lancet 2001: negative correlation between coronary heart disease mortality and level of vitamin C in bloodstream (controlling for
age, gender, blood pressure, diabetes, and smoking)
\onslide<2>
Lancet 2002: no effect of vitamin C on mortality in controlled placebo trial (controlling for nothing)
\onslide<3>
Lancet 2003: comparing among individuals with the same age, gender, blood pressure, diabetes, and smoking, those with higher vitamin C levels have lower levels of obesity, lower levels of alcohol consumption, are less likely to grow up in working class, etc.
\end{overprint}

}




\begin{frame}{Observational Studies}

\begin{itemize}
\itemsep1pt\parskip0pt\parsep0pt
\item
  Randomization forms gold standard for causal inference, because it
  balances \alert{observed} and \alert{unobserved} confounders\medskip
\item
  Cannot always randomize so we do observational studies, where we
  \alert{adjust} for the \alert{observed covariates} and \alert{hope}
  that unobservables are balanced\medskip
\item
  Better than hoping: \alert{design} observational study to approximate
  an experiment

  \begin{itemize}
  \itemsep1pt\parskip0pt\parsep0pt
  \item
    ``The planner of an observational study should always ask himself:
    How would the study be conducted if it were possible to do it by
    controlled experimentation'' (Cochran 1965)
  \end{itemize}
\end{itemize}

\end{frame}

\section{What Makes a Good Observational
Study?}\label{what-makes-a-good-observational-study}

\begin{frame}{The Good, the Bad, and the Ugly}

\textbf{Treatments, Covariates, Outcomes}

\begin{itemize}
\itemsep1pt\parskip0pt\parsep0pt
\item
  \alert{Randomized Experiment}: Well-defined treatment, clear
  distinction between covariates and outcomes, control of assignment mechanism\medskip
\item
  \alert{Better Observational Study}: Well-defined treatment, clear
  distinction between covariates and outcomes, precise knowledge of assignment mechanism\

  \begin{itemize}
  \itemsep1pt\parskip0pt\parsep0pt
  \item
    Can convincingly answer the following question: Why do two units who
    are identical on measured covariates receive different treatments?
  \end{itemize}
\item
  \alert{Poorer Observational Study}: Hard to say when treatment began
  or what the treatment really is. Distinction between covariates and
  outcomes is blurred, so problems that arise in experiments seem to be
  avoided but are in fact just ignored. No precise knowledge of assignment mechanism.
\end{itemize}

\end{frame}



\begin{frame}{The Good, the Bad, and the Ugly}

\textbf{How were treatments assigned?}

\begin{itemize}
\itemsep1pt\parskip0pt\parsep0pt
\item
  \alert{Randomized Experiment}: Random assignment \medskip
\item
  \alert{Better Observational Study}: Assignment is not random, but
  circumstances for the study were chosen so that treatment seems
  haphazard, or at least not obviously related to potential outcomes
  (sometimes we refer to these as natural or quasi-experiments) \medskip
\item
  \alert{Poorer Observational Study}: No attention given to assignment
  process, units self-select into treatment based on potential outcomes
\end{itemize}

\end{frame}

\begin{frame}{The Good, the Bad, and the Ugly}

\textbf{What is the problem with purely cross-sectional data?}

\begin{itemize}
\itemsep1pt\parskip0pt\parsep0pt
\item
  Difficult to know what is pre or post treatment. \medskip
\item
  Many important confounders will be affected by the treatment and including these ``bad controls'' induces post-treatment bias. \medskip
\item
  But if you do not condition on the confounders that are
  post-treatment, then often only left with a limited set of covariates
  such as socio-demographics.
\end{itemize}

\end{frame}

\begin{frame}{The Good, the Bad, and the Ugly}

\textbf{Were treated and controls comparable?}

\begin{itemize}
\itemsep1pt\parskip0pt\parsep0pt
\item
  \alert{Randomized Experiment}: Balance table for observables. \medskip
\item
  \alert{Better Observational Study}: Balance table for observables.
  Ideally sensitivity analysis for unobservables. \medskip
\item
  \alert{Poorer Observational Study}: No direct assessment of
  comparability is presented.
\end{itemize}

\end{frame}

\begin{frame}{The Good, the Bad, and the Ugly}

\textbf{Eliminating plausible alternatives to treatment effects?}

\begin{itemize}
\itemsep1pt\parskip0pt\parsep0pt
\item
  \alert{Randomized Experiment}: List plausible alternatives and
  experimental design includes features that shed light on these
  alternatives (e.g.~placebos). Report on potential attrition and
  non-compliance. \medskip
\item
  \alert{Better Observational Study}: List plausible alternatives and
  study design includes features that shed light on these alternatives
  (e.g.~multiple control groups, longitudinal covariate data, etc.).
  Requires more work than in experiment since there are usually many
  more alternatives. \medskip
\item
  \alert{Poorer Observational Study}: Alternatives are mentioned in
  discussion section of the paper.
\end{itemize}

\end{frame}

\begin{frame}{Good Observational Studies}

\alert{Design features} we can use to handle unobservables:

\begin{itemize}
\itemsep1pt\parskip0pt\parsep0pt
\item
  Design comparisons so that unobservables are likely to be balanced
  (e.g.~sub-samples, groups where treatment assignment was accidental,
  etc.) \medskip
\item
  Unobservables may differ, but comparisons that are unaffected by
  differences in time-invariant unobservables \medskip
\item
  Instrumental variables, if applied correctly \medskip
\item
  Multiple control groups that are known to differ on unobservables
  \medskip
\item
  Sensitivity analysis and bounds \medskip
\end{itemize}

\end{frame}

\begin{frame}{Good Observational Studies}

\begin{enumerate}
\def\labelenumi{\arabic{enumi}.}
\itemsep1pt\parskip0pt\parsep0pt
\item
  An observational study should be structured to resemble an experiment.\medskip
\item
  Adjustment for observed covariates should be simple, transparent, and
  convincing.\medskip
\item
  The most plausible alternatives to the treatment effect should be
  anticipated and addressed.\medskip
\item
  The analysis should address possible biases from unmeasured
  covariates.
%\item There should be a plan for a primary analysis.
\end{enumerate}

\end{frame}

\begin{frame}{Class Size on Student Achievements}

\begin{figure}[ht] \centering
    \includegraphics[width = .75 \linewidth]{images/ang1.pdf}
\end{figure}

Angrist and Lavy (1999)

\end{frame}

\begin{frame}{Class Size on Student Achievements}

\begin{figure}[ht] \centering
    \includegraphics[width = .75 \linewidth]{images/ang2.pdf}
\end{figure}

Angrist and Lavy (1999)

\end{frame}

\begin{frame}{Seat Belts on Fatality Rates}

\begin{figure}[ht] \centering
    \includegraphics[width = .95 \linewidth]{images/seat_belts.pdf}
\end{figure}

Evans (1986)

\end{frame}

\begin{frame}{Persuasive Effect of Endorsement Changes on Labour Vote}

\begin{figure}[ht] \centering
    \includegraphics[width = .48 \linewidth]{images/LaddLenz1.pdf}
\end{figure}

Ladd and Lenz (1999)

\end{frame}

\begin{frame}{Persuasive Effect of Endorsement Changes on Labour Vote}

\begin{figure}[ht] \centering
    \includegraphics[width = .66 \linewidth]{images/LaddLenz2.pdf}
\end{figure}

Ladd and Lenz (1999)

\end{frame}

\section{Removing Bias by
Conditioning}\label{removing-bias-by-conditioning}

\begin{frame}{Adjustment for Observables in Observational Studies}

\begin{itemize}
\itemsep1pt\parskip0pt\parsep0pt
\item
  Subclassification\bigskip
\item
  Matching \bigskip
\item
  Propensity Score Methods\bigskip
\item
  Regression
\end{itemize}

\end{frame}

\begin{frame}{Smoking and Mortality (Cochran (1968))}

\begin{center}
 \begin{tabular}{lcccccc}
 \multicolumn{7}{c}{\small\sc Table 1}\\
 \multicolumn{7}{c}{\footnotesize\sc Death Rates per
 1,000 Person-Years}\vspace*{0.1cm}\\\hline\hline\\
 Smoking group&&Canada&&U.K.&&U.S.\\\hline\\
 Non-smokers&&20.2&&11.3&&13.5\\
 Cigarettes&&20.5&&14.1&&13.5\\
 Cigars/pipes&&35.5&&20.7&&17.4\\\hline\\
 \end{tabular}
\end{center}

\end{frame}

\begin{frame}{Smoking and Mortality (Cochran (1968))}

\begin{center}
 \begin{tabular}{lcccccc}
 \multicolumn{7}{c}{\small\sc Table 2}\\
 \multicolumn{7}{c}{\footnotesize\sc Mean Ages, Years}
 \vspace*{0.1cm}\\\hline\hline\\
 Smoking group&&Canada&&U.K.&&U.S.\\\hline\\
 Non-smokers&&54.9&&49.1&&57.0\\
 Cigarettes&&50.5&&49.8&&53.2\\
 Cigars/pipes&&65.9&&55.7&&59.7\\\hline\\
 \end{tabular}
\end{center}

\end{frame}

\begin{frame}{Subclassification}

To control for differences in age, we would like to compare different
smoking-habit groups with the same age distribution\bigskip

One possibility is to use subclassification:

\begin{itemize}
\itemsep1pt\parskip0pt\parsep0pt
\item
  for each country, divide each group into different age subgroups
\item
  calculate death rates within age subgroups
\item
  average within age subgroup death rates using fixed weights (e.g.
  number of cigarette smokers)
\end{itemize}

\end{frame}

\begin{frame}{Subclassification: Example}

\begin{center}
\begin{tabular}{|c|c|c|c|}
\hline
\multicolumn{ 1}{|c|}{} & \multicolumn{ 1}{c|}{Death Rates  } & \# Pipe- & \# Non- \\

\multicolumn{ 1}{|c|}{} & \multicolumn{ 1}{c|}{Pipe Smokers} &  Smokers  &  Smokers \\
\hline
Age 20 - 50  &         15 &         11 &         29 \\
\hline
Age 50 - 70  &         35 &         13 &          9 \\
\hline
 Age + 70  &         50 &         16 &          2 \\
\hline
    Total  &            &         40 &         40 \\
\hline
\end{tabular}
\end{center}

What is the average death rate for Pipe Smokers? \pause

$15\cdot(11/40)+35\cdot(13/40)+50\cdot(16/40)=35.5$

\end{frame}

\begin{frame}{Subclassification: Example}

\begin{center}
\begin{tabular}{|c|c|c|c|}
\hline
\multicolumn{ 1}{|c|}{} & \multicolumn{ 1}{c|}{Death Rates  } & \# Pipe- & \# Non- \\

\multicolumn{ 1}{|c|}{} & \multicolumn{ 1}{c|}{Pipe Smokers} &  Smokers  &  Smokers \\
\hline
Age 20 - 50  &         15 &         11 &         29 \\
\hline
Age 50 - 70  &         35 &         13 &          9 \\
\hline
 Age + 70  &         50 &         16 &          2 \\
\hline
    Total  &            &         40 &         40 \\
\hline
\end{tabular}
\end{center}

What is the average death rate for Pipe Smokers if they had same age
distribution as Non-Smokers? \pause

$15\cdot(29/40)+35\cdot(9/40)+50\cdot(2/40)=21.2$

\end{frame}

\begin{frame}{Smoking and Mortality (Cochran (1968))}

\begin{center}
 \begin{tabular}{lcccccc}
 \multicolumn{7}{c}{\small\sc Table 3}\\
 \multicolumn{7}{c}{\footnotesize\sc Adjusted Death Rates using 3 Age groups}
 \vspace*{0.1cm}\\\hline\hline\\
 Smoking group&&Canada&&U.K.&&U.S.\\\hline\\
 Non-smokers&&20.2&&11.3&&13.5\\
 Cigarettes&&28.3&&12.8&&17.7\\
 Cigars/pipes&&21.2&&12.0&&14.2\\\hline\\
 \end{tabular}
\end{center}

\end{frame}

\section{Identification Under Selection on
Observables}\label{identification-under-selection-on-observables}

\begin{frame}{Identification Under Selection on Observables}

\begin{iass}
\begin{enumerate}\small
\item $(Y_1,Y_0) \indep D |X$ (selection on observables)
\item $0<\Pr(D=1|X)<1$ with probability one (common support)
\end{enumerate}
\end{iass}

\vspace{-.2in}

\begin{overprint}
\onslide<1>
\begin{ires}\small
Given selection on observables we have
\begin{eqnarray}
E[Y_1-Y_0|X]&=&E[Y_1-Y_0|X,D=1]\nonumber\\
           &=& E[Y|X,D=1]-E[Y|X,D=0]\nonumber
\end{eqnarray}
Therefore, under the common support condition:
\begin{eqnarray}
\tau_{ATE}&=&E[Y_1-Y_0] = \int E[Y_1-Y_0|X]\,dP(X)\nonumber\\
           &=& \int \big(E[Y|X,D=1]-E[Y|X,D=0]\big)\,dP(X)\nonumber
\end{eqnarray}
\end{ires}
\onslide<2>
\begin{ires}\small
Similarly,
\begin{eqnarray}
\tau_{ATT}&=&E[Y_1-Y_0|D=1]\nonumber\\
&=&\int \big(E[Y|X,D=1]-E[Y|X,D=0]\big)\,dP(X|D=1)\nonumber
\end{eqnarray}
To identify $\tau_{ATT}$ the selection on observables
and common support conditions can be relaxed to: \begin{itemize}
\item $Y_0\indep D | X$ (SOO for Controls)
\item $\Pr(D=1|X)<1$ (Weak Overlap)
\end{itemize}
\end{ires}
\end{overprint}

\end{frame}

\begin{frame}{Identification Under Selection on Observables}

\begin{overprint}
\onslide<1>
\begin{small}
\begin{tabular}{c|c|c|c|c}
 &Potential Outcome & Potential Outcome &   & \\
unit &under Treatment  & under Control &   &  \\
\hline
   i  & $Y_{1i}$  & $Y_{0i}$ &    $D_i$ &  $X_i$ \\
\hline
         1 & \multirow{2}{*}{$E[Y_1|X=0,D=1]$ } & \multirow{2}{*}{\textcolor{red}{$E[Y_0|X=0,D=1]$} } &                    1 &          0 \\
         2 & \multicolumn{ 1}{c|}{} & \multicolumn{ 1}{c|}{} &                    1 &          0 \\
\hline
         3 & \multirow{2}{*}{\textcolor{red}{$E[Y_1|X=0,D=0]$}} & \multirow{2}{*}{$E[Y_0|X=0,D=0]$ } &                   0 &          0 \\
         4 & \multicolumn{ 1}{c|}{} & \multicolumn{ 1}{c|}{} &                    0 &          0 \\
\hline
         5 & \multirow{2}{*}{$E[Y_1|X=1,D=1]$ } & \multirow{2}{*}{\textcolor{red}{$E[Y_0|X=1,D=1]$}} &                    1 &          1 \\
         6 & \multicolumn{ 1}{c|}{} & \multicolumn{ 1}{c|}{} &                    1 &          1 \\
\hline
         7 & \multirow{2}{*}{\textcolor{red}{$E[Y_1|X=1,D=0]$}} & \multirow{2}{*}{$E[Y_0|X=1,D=0]$} &                    0 &          1 \\
         8 & \multicolumn{ 1}{c|}{} & \multicolumn{ 1}{c|}{} &                   0 &          1 \\
\hline
\end{tabular}
\end{small}
\onslide<2>
\begin{small}
\begin{tabular}{c|c|c|c|c}
 &Potential Outcome & Potential Outcome &   & \\
unit &under Treatment  & under Control &   &  \\
\hline
   i  & $Y_{1i}$  & $Y_{0i}$ &    $D_i$ &  $X_i$ \\
\hline
         1 & \multirow{2}{*}{$E[Y_1|X=0,D=1]$ } & \multicolumn{ 1}{c|}{\textcolor{red}{$E[Y_0|X=0,D=1]$}= } &                    1 &          0 \\
         2 & \multicolumn{ 1}{c|}{} & \multicolumn{ 1}{l|}{$E[Y_0|X=0,D=0]$} &                    1 &          0 \\
\hline
         3 & \multirow{2}{*}{\textcolor{red}{$E[Y_1|X=0,D=0]$}} & \multirow{2}{*}{$E[Y_0|X=0,D=0]$ } &                   0 &          0 \\
         4 & \multicolumn{ 1}{c|}{} & \multicolumn{ 1}{c|}{} &                    0 &          0 \\
\hline
         5 & \multirow{2}{*}{$E[Y_1|X=1,D=1]$ } & \multicolumn{ 1}{c|}{\textcolor{red}{$E[Y_0|X=1,D=1]$}=} &                    1 &          1 \\
         6 & \multicolumn{ 1}{c|}{} & \multicolumn{ 1}{l|}{$E[Y_0|X=1,D=0]$} &                    1 &          1 \\
\hline
         7 & \multirow{2}{*}{\textcolor{red}{$E[Y_1|X=1,D=0]$}} & \multirow{2}{*}{$E[Y_0|X=1,D=0]$} &                    0 &          1 \\
         8 & \multicolumn{ 1}{c|}{} & \multicolumn{ 1}{c|}{} &                   0 &          1 \\
\hline
\end{tabular}
\end{small}\bigskip\\
$(Y_1,Y_0) \indep D |X$ implies that we conditioned on all confounders. The treatment is randomly assigned within each stratum of $X$:\vspace{-.1in}
\begin{eqnarray}
E[Y_0|X=0,D=1] &=& E[Y_0|X=0,D=0]\,\, \mbox{and}\nonumber\\
 E[Y_0|X=1,D=1] &=& E[Y_0|X=1,D=0]\nonumber
\end{eqnarray}
\onslide<3>
\begin{small}
\begin{tabular}{c|c|c|c|c}
 &Potential Outcome & Potential Outcome &   & \\
unit &under Treatment  & under Control &   &  \\

\hline
   i  & $Y_{1i}$  & $Y_{0i}$ &    $D_i$ &  $X_i$ \\
\hline
         1 & \multirow{2}{*}{$E[Y_1|X=0,D=1]$ } & \multicolumn{ 1}{c|}{\textcolor{red}{$E[Y_0|X=0,D=1]$}= } &                    1 &          0 \\
         2 & \multicolumn{ 1}{c|}{} & \multicolumn{ 1}{l|}{$E[Y_0|X=0,D=0]$} &                    1 &          0 \\
\hline
         3 & \multicolumn{ 1}{c|}{\textcolor{red}{$E[Y_1|X=0,D=0]=$}} & \multirow{2}{*}{$E[Y_0|X=0,D=0]$ } &                   0 &          0 \\
         4 & \multicolumn{ 1}{l|}{$E[Y_1|X=0,D=1]$} & \multicolumn{ 1}{c|}{} &                    0 &          0 \\
\hline
         5 & \multirow{2}{*}{$E[Y_1|X=1,D=1]$ } & \multicolumn{ 1}{c|}{\textcolor{red}{$E[Y_0|X=1,D=1]$}=} &                    1 &          1 \\
         6 & \multicolumn{ 1}{c|}{} & \multicolumn{ 1}{l|}{$E[Y_0|X=1,D=0]$} &                    1 &          1 \\
\hline
         7 &\multicolumn{ 1}{c|}{\textcolor{red}{$E[Y_1|X=1,D=0]=$}} & \multirow{2}{*}{$E[Y_0|X=1,D=0]$} &                    0 &          1 \\
         8 & \multicolumn{ 1}{l|}{$E[Y_1|X=1,D=1]$ } & \multicolumn{ 1}{c|}{} &                   0 &          1 \\
         \hline
\end{tabular}
\end{small}\bigskip\\
$(Y_1,Y_0) \indep D |X$ also implies
\begin{eqnarray}
E[Y_1|X=0,D=1] &=& E[Y_1|X=0,D=0]\,\, \mbox{and}\nonumber\\
 E[Y_1|X=1,D=1] &=& E[Y_1|X=1,D=0]\nonumber
\end{eqnarray}
\end{overprint}

\end{frame}

%\subsection{Post-Treatment Bias}\label{post-treatment-bias}
%
%\begin{frame}{Post-Treatment Bias}
%
%\textbf{What are the dangers of conditioning on a post-treatment
%variable? }
%
%\begin{itemize}
%\itemsep1pt\parskip0pt\parsep0pt
%\item
%  Imagined scenario: You are interested in estimating the effect of
%  private school versus public high school education on test scores in
%  grade 12: $\tau = E[ Y_1 - Y_0]$.
%\item
%  You observe test scores in grade 12 and grade 10.
%\item
%  A reviewer argues that you should control for grade 10 test scores, in
%  order to account for baseline differences in test scores.
%\item
%  When is adjusting for a post-treatment variable justified?
%
%  \begin{itemize}
%  \itemsep1pt\parskip0pt\parsep0pt
%  \item
%    Rosenbaum (1984) provides the classic analysis: ``The Consequences
%    of Adjustment for a Concomitant Variable That has Been Affected by
%    Treatment''. \emph{Journal of the Royal Statistics Society, Series
%    A.} 147: 656:666.
%  \end{itemize}
%\end{itemize}
%
%\end{frame}
%
%\begin{frame}{Post-Treatment Bias}
%
%\begin{itemize}
%\itemsep1pt\parskip0pt\parsep0pt
%\item
%  10th grade test scores can be written as follows:
%  $S_i =     D_iS_1 + (1- D) S_0$ \pause
%\item
%  The reviewer wants you to estimate the following quantity:
%  \[\tilde \tau = E_S\{E[Y|D = 1, S = s\}] - E[Y|D = 0, S = s]\}\]
%\item
%  This is an estimator for the average treatment effect controlling for
%  the post-treatment variable.
%\end{itemize}
%
%\end{frame}
%
%\begin{frame}{Post-Treatment Bias}
%
%Useful to specify the ``Net Treatment Difference'':
%\[ \tilde v = \E\{E[Y_1 | S_1 = s] - E[Y_0 | S_0 = s] \}\]
%
%This would compare private school students with public school students
%with identical 10th grade test scores.
%
%\begin{itemize}
%\itemsep1pt\parskip0pt\parsep0pt
%\item
%  When does $\tau > 0$ and $\tilde v = 0$ suggest about $S$?
%
%  \begin{itemize}
%  \itemsep1pt\parskip0pt\parsep0pt
%  \item
%    \pause Positive effect is transmitted through $S$.
%  \end{itemize}
%\item
%  When does $\tau >0 $ and $\tilde v > 0$ suggest about $S$?
%
%  \begin{itemize}
%  \itemsep1pt\parskip0pt\parsep0pt
%  \item
%    \pause Positive effect is transmitted through variables other than
%    $S$.
%  \end{itemize}
%\item
%  When does $\tau > 0$ and $\tilde v < 0$ suggest about the effect of
%  the treatment?
%
%  \begin{itemize}
%  \itemsep1pt\parskip0pt\parsep0pt
%  \item
%    \pause Positive effects over all, but some negative effects.
%  \end{itemize}
%\end{itemize}
%
%\end{frame}
%
%\begin{frame}{Post-Treatment Bias}
%
%What is the bias when conditioning on a post-treatment variable? (Assume
%no confounding, i.e. ``as if'' randomization)
%\[ \tilde \tau - \tau = (\tilde \tau - \tilde v) + (\tilde v -
%\tau)  \]
%
%\begin{itemize}
%\itemsep1pt\parskip0pt\parsep0pt
%\item
%  The first term $(\tilde \tau - \tilde v)$ will be 0 under random
%  assignment or ``as if'' random assignment.
%\item
%  Bias from the second term will be present when the treatment affects
%  $S$ and $S$ affects $Y$ for any unit. \pause
%\item
%  Bias will be absent if the treatment variable has no effect on $S$ for
%  all $i$ or the treatment has no effect on $Y$ through $S$ for all $i$
%  (no indirect effects). \pause
%\item
%  Overall lesson: \textbf{do not condition on post-treatment
%      variables}, unless you are confident treatment has no effect on
%  the post-treatment variable.
%\end{itemize}
%
%\end{frame}

\subsection{Subclassification}\label{subclassification-1}

\begin{frame}{Subclassification Estimator}

\begin{ires}
\begin{eqnarray}\small
\tau_{ATE}&=& \int \big(E[Y|X,D=1]-E[Y|X,D=0]\big)\,dP(X) \nonumber\\
\tau_{ATT}&=&\int \big(E[Y|X,D=1]-E[Y|X,D=0]\big)\,dP(X|D=1) \nonumber
\end{eqnarray}
\end{ires}

Assume $X$ takes on $K$ different cells $\{X^1,..., X^k,...,X^K\}$. Then
the analogy principle suggests estimators: \pause

\begin{eqnarray*}
\widehat{\tau}_{ATE}= \sum_{k=1}^K \big(
\bar{Y}_1^k - \bar{Y}_0^k
\big)\cdot
\left(
\frac{N^k}{N}
\right);\,\,
\widehat{\tau}_{ATT}= \sum_{k=1}^K \big(
\bar{Y}_1^k - \bar{Y}_0^k
\big)\cdot
\left(
\frac{N_1^k}{N_1}
\right)
\end{eqnarray*}

\begin{itemize}
\itemsep1pt\parskip0pt\parsep0pt
\item
  $N^k$ is \# of obs. and $N_1^k$ is \# of treated obs. in cell $k$
\item
  $\bar{Y}_1^k$ is mean outcome for the treated in cell $k$
\item
  $\bar{Y}_0^k$ is mean outcome for the untreated in cell $k$
\end{itemize}

\end{frame}

\begin{frame}{Subclassification by Age ($K = 2$)}

\begin{tabular}{c|c|c|c|c|c}
           & Death Rate & Death Rate &            &          \# &         \#  \\
       $X_k$ &    Smokers & Non-Smokers &      Diff. &    Smokers &       Obs. \\
       \hline
       Old &         28 &         24 &         4 &          3 &         10 \\
       \hline
     Young &         22 &         16 &         6 &          7 &         10 \\
     \hline
     Total &            &            &            &         10 &         20 \\
\end{tabular}

What is
$\widehat{\tau}_{ATE}= \sum_{k=1}^K \big( \bar{Y}_1^k - \bar{Y}_0^k \big)\cdot \left( \frac{N^k}{N} \right)$?
\pause

$\widehat{\tau}_{ATE}=4\cdot(10/20)+6\cdot(10/20)=5$

\end{frame}

\begin{frame}{Subclassification by Age ($K = 2$)}

\begin{tabular}{c|c|c|c|c|c}
           & Death Rate & Death Rate &            &          \# &         \#  \\
       $X_k$ &    Smokers & Non-Smokers &      Diff. &    Smokers &       Obs. \\
       \hline
       Old &         28 &         24 &         4 &          3 &         10 \\
       \hline
     Young &         22 &         16 &         6 &          7 &         10 \\
     \hline
     Total &            &            &            &         10 &         20 \\
\end{tabular}

What is
$\widehat{\tau}_{ATT}= \sum_{k=1}^K \big( \bar{Y}_1^k - \bar{Y}_0^k \big)\cdot \left( \frac{N_1^k}{N_1} \right)$?
\pause

$\widehat{\tau}_{ATT}=4\cdot(3/10)+6\cdot(7/10)=5.4$

\end{frame}

\begin{frame}{Subclassification by Age and Gender ($K = 4$)}

\begin{tabular}{l|c|c|c|c|c}
           & Death Rate & Death Rate &            &         \# &         \#  \\
       $X_k$ &    Smokers & Non-Smokers &      Diff. &    Smokers &       Obs. \\
       \hline
 Old, Male &         28 &         22 &          4 &          3 &          7 \\
 \hline
Old, Female &            &         24 &            &          0 &          3 \\
\hline
Young, Male &         21 &         16 &          5 &          3 &          4 \\
\hline
Young, Female &         23 &         17 &          6 &          4 &          6 \\
\hline
     Total &            &            &            &         10 &         20 \\
\end{tabular}

What is
$\widehat{\tau}_{ATE}= \sum_{k=1}^K \big( \bar{Y}_1^k - \bar{Y}_0^k \big)\cdot \left( \frac{N^k}{N} \right)$?
\pause

\alert{Not identified!}

\end{frame}

\begin{frame}{Subclassification by Age and Gender ($K = 4$)}

\begin{tabular}{l|c|c|c|c|c}
           & Death Rate & Death Rate &            &         \# &         \#  \\
       $X_k$ &    Smokers & Non-Smokers &      Diff. &    Smokers &       Obs. \\
       \hline
 Old, Male &         28 &         22 &          4 &          3 &          7 \\
 \hline
Old, Female &            &         24 &            &          0 &          3 \\
\hline
Young, Male &         21 &         16 &          5 &          3 &          4 \\
\hline
Young, Female &         23 &         17 &          6 &          4 &          6 \\
\hline
     Total &            &            &            &         10 &         20 \\
\end{tabular}

What is
$\widehat{\tau}_{ATT}= \sum_{k=1}^K \big( \bar{Y}_1^k - \bar{Y}_0^k \big)\cdot \left( \frac{N_1^k}{N_1} \right)$?
\pause

$\widehat{\tau}_{ATT}= 4\cdot(3/10)+5\cdot(3/10)+6\cdot(4/10)=5.1$

\end{frame}

%\begin{frame}{Fisher on Smoking}
%
%R.A. Fisher:
%
%\begin{quote}
%The causes of smoking cigarettes may be studied among your friends, to
%some extent, and I think you will agree that a slight cause of
%irritation---a slight disappointment, an unexpected delay, some sort of
%a mild rebuff, a frustration---are {[}sic{]} commonly accompanied by
%pulling out a cigarette and getting a little compensation for life's
%minor ills in that way. And so, anyone suffering from a chronic
%inflammation in part of the body (something that does not give rise to
%conscious pain) is not unlikely to be associated with smoking more
%frequently, or smoking rather than not smoking. \ldots{} And to take the
%poor chap's cigarettes away from him would be rather like taking away
%his white stick from a blind man.
%\end{quote}
%
%\end{frame}
%
%\begin{frame}{Example of Subclassification}
%
%Maya Sen, in ``Exploring the Role of Race and Gender in the Subprime
%Lending Crisis'', asks if there is evidence of discriminatory lending of
%subprime loans and uses a database of all real estate transactions HMDA
%database:
%
%\begin{figure}[ht] \centering
%    \includegraphics[width = .8 \linewidth]{images/sen.png}
%\end{figure}
%
%\end{frame}

\subsection{Matching}\label{matching}

\begin{frame}{}

\centering

\fbox{Matching is Not an Identification Strategy}

\end{frame}

%\begin{frame}{Lyall (2009)}
%
%\begin{figure}[ht] \centering
%    \includegraphics[width = .8 \linewidth]{images/lyall_map.png}
%\end{figure}
%
%\end{frame}
%
%\begin{frame}{Identification Strategy: Doctrine and Drunks}
%
%\textbf{``Harassment and Interdiction'' Doctrine}: ``barrages at random
%intervals and of varying duration on random days'' \pause \bigskip
%
%\textbf{Drunks}: ``As one Russian commander put it, solders `get drunk
%as pigs, lob a few shells, claim combat pay and get drunk again'.''"
%
%\end{frame}
%
%\begin{frame}{Balance Checks}
%
%\begin{figure}[ht] \centering
%    \includegraphics[width = .99 \linewidth]{images/lyall_balance.png}
%\end{figure}
%
%\end{frame}

%\begin{frame}{Gilligan and Sargenti (2008)}
%
%\begin{figure}[ht] \centering
%    \includegraphics[width = .99 \linewidth]{images/gilligan_sargenti_balance.png}
%\end{figure}
%
%\end{frame}

%\begin{frame}{Gilligan and Sargenti (2008)}
%
%\begin{figure}[ht] \centering
%    \includegraphics[width = .99 \linewidth]{images/gilligan_sargenti_matched_pairs.png}
%\end{figure}
%
%\end{frame}



\begin{frame}{Matching}

When $X$ is continuous we can estimate $\tau_{ATT}$ by ``imputing''
the missing potential outcome of each treated unit using the observed
outcome from the ``closest'' control unit: \[
\widehat{\tau}_{ATT}= \frac{1}{N_1} \sum_{D_i=1}
\big(Y_i - Y_{j(i)}\big)
\] where $Y_{j(i)}$ is the outcome of an untreated observation such that
$X_{j(i)}$ is the \alert{closest} value to $X_i$ among the untreated
observations. \pause

We can also use the average for $M$ closest matches: \[
\widehat{\tau}_{ATT}= \frac{1}{N_1} \sum_{D_i=1} \left\{ Y_i -
\left(\frac{1}{M}\sum_{m=1}^M Y_{j_m(i)},\right)\right\}
\] Works well when we can find good matches for each treated unit

\end{frame}

\begin{frame}{Matching: Example with a Single $X$}

\begin{tabular}{c|c|c|c|c}
 &Potential Outcome & Potential Outcome &   & \\
unit &under Treatment  & under Control &   &  \\

\hline
   i  & $Y_{1i}$  & $Y_{0i}$ &    $D_i$ &  $X_i$ \\
\hline
         1 & \multicolumn{ 1}{c|}{6} & \multicolumn{ 1}{c|}{\textcolor{red}{?}} &                    1 &          3 \\
         2 & \multicolumn{ 1}{c|}{1} & \multicolumn{ 1}{c|}{\textcolor{red}{?}} &                    1 &          1 \\
         3 & \multicolumn{ 1}{c|}{0} & \multicolumn{ 1}{c|}{\textcolor{red}{?}} &                    1 &          10 \\
\hline
         4 & \multicolumn{ 1}{c|}{} & \multicolumn{1}{c|}{0} &                    0 &          2 \\
         5 & \multicolumn{ 1}{c|}{} & \multicolumn{1}{c|}{9} &                    0 &          3 \\
         6 & \multicolumn{ 1}{c|}{} & \multicolumn{1}{c|}{1} &                    0 &          -2 \\
         7 & \multicolumn{ 1}{c|}{} & \multicolumn{1}{c|}{1} &                    0 &          -4 \\
         \hline
\end{tabular}

What is
$\widehat{\tau}_{ATT}= \frac{1}{N_1} \sum_{D_i=1} \big(Y_i - Y_{j(i)}\big)$?
\pause

Match and plugin in

\end{frame}

\begin{frame}{Matching: Example with a Single $X$}

\begin{tabular}{c|c|c|c|c}
 &Potential Outcome & Potential Outcome &   & \\
unit &under Treatment  & under Control &   &  \\

\hline
   i  & $Y_{1i}$  & $Y_{0i}$ &    $D_i$ &  $X_i$ \\
\hline
         1 & \multicolumn{ 1}{c|}{6} & \multicolumn{ 1}{c|}{\textcolor{blue}{9}} &                    1 &          \textcolor{blue}{3} \\
         2 & \multicolumn{ 1}{c|}{1} & \multicolumn{ 1}{c|}{\textcolor{cyan}{0}} &                    1 &          \textcolor{cyan}{1} \\
         3 & \multicolumn{ 1}{c|}{0} & \multicolumn{ 1}{c|}{\textcolor{blue}{9}} &                    1 &          \textcolor{blue}{10} \\
\hline
         4 & \multicolumn{ 1}{c|}{} & \multicolumn{1}{c|}{\textcolor{cyan}{0}} &                    0 &          \textcolor{cyan}{2} \\
         5 & \multicolumn{ 1}{c|}{} & \multicolumn{1}{c|}{\textcolor{blue}{9}} &                    0 &          \textcolor{blue}{3} \\
         6 & \multicolumn{ 1}{c|}{} & \multicolumn{1}{c|}{1} &                    0 &          -2 \\
         7 & \multicolumn{ 1}{c|}{} & \multicolumn{1}{c|}{1} &                    0 &          -4 \\
         \hline
\end{tabular}

What is
$\widehat{\tau}_{ATT}= \frac{1}{N_1} \sum_{D_i=1} \big(Y_i - Y_{j(i)}\big)$?
\pause

$\widehat{\tau}_{ATT}=1/3\cdot(6-\textcolor{blue}{9})+1/3\cdot(1-\textcolor{cyan}{0})+1/3\cdot(0-\textcolor{blue}{9})=-3.7$

\end{frame}

\begin{frame}{Matching Distance Metric}
\scriptsize
``Closeness'' is often defined by a \alert{distance metric}. Let
$X_i=(X_B,X_B,...,X_{ik})'$ and $X_j=(X_{j1},X_{j2},...,X_{jk})'$
be the covariate vectors for $i$ and $j$.\\\medskip A commonly used distance is
the \alert{Mahalanobis distance}: \[
MD(X_i,X_j)=\sqrt{(X_i-X_j)'\Sigma^{-1}(X_i-X_j)}
\] where $\Sigma$ is the Variance-Covariance-Matrix so the distance metric is scale-invariant and takes into account the correlations. For an exact match
$MD(X_i,X_j)=0$. 

Other distance metrics can be used, for example Stata's \texttt{nnmatch} by default uses the diagonal matrix of the inverse of the covariate variances (normalized Euclidean distance):
\[
StataD(X_i,X_j)=\sqrt{(X_i-X_j)'diag(\Sigma_X^{-1})(X_i-X_j)}
\]

In \texttt{R}, Genetic matching uses (\texttt{GenMatch(Matching)}): \[
GeneticD(X_i,X_j) = \sqrt{(X_i - X_j)'(S^{-1/2})'\, W\, S^{-1/2} (X_i - X_j)}
\] where $W$ is a $(k \times k)$ positive definite weight matrix with
zeros in off-diagonals.

\end{frame}

%\begin{frame}{Mahalanobis Distance}
%
%\begin{figure}[ht] \centering
%    \includegraphics[width = 1 \linewidth]{images/mahal.png}
%\end{figure}
%
%\end{frame}
%
%\begin{frame}{Mahalanobis Distance}
%
%\begin{figure}[ht] \centering
%    \includegraphics[width = .8 \linewidth]{images/choleskytransform.png}
%\end{figure}
%
%\end{frame}

\begin{frame}{Mahalanobis Distance: Example}

\begin{table}[!h]
 \begin{center}
 \begin{tabular}{ccc}
\multicolumn{1}{l}{}&
\multicolumn{1}{c}{$X_1$}&
\multicolumn{1}{c}{$X_2$}
\\ \hline
Treated &$0$&$  0$\\
Control A &$ 2$&$2$\\
Control B &$ 1.8$&$0$\\
\hline
\end{tabular}
\end{center}
\end{table}

$X_T=\left(        \begin{array}{cc}          0 & 0 \\        \end{array}      \right)' $
$\quad X_A=\left(                 \begin{array}{cc}                   2 & 2 \\                 \end{array}               \right)' $
$\quad X_B=\left(                 \begin{array}{cc}                   1.8 &  0 \\                 \end{array}               \right)' $\\\bigskip

$\quad\Sigma=\left(                \begin{array}{ccc}                   & X_1 & X_2 \\                  X_1 & 1 &  .9 \\                  X_2 & .9 & 1\\                \end{array}              \right)$\\\bigskip

Which control is closer?

\end{frame}

\begin{frame}{Mahalanobis Distance}
\scriptsize
$X_T=\left(        \begin{array}{cc}          0 & 0 \\        \end{array}      \right)' $
$\,\, X_A=\left(                 \begin{array}{cc}                   2 &  2 \\                 \end{array}               \right)' $
$\,\, X_B=\left(                 \begin{array}{cc}                   1.8 &  0 \\                 \end{array}               \right)' $
$\,\,\Sigma=\left(                \begin{array}{ccc}           & X_1 & X_2 \\                  X_1 & 1 &  .9 \\                  X_2 & .9 & 1\\                \end{array}              \right)$

\begin{align*}
MD(X_A,X_{T}) &= \sqrt{
(X_A-X_T)'\Sigma^{-1}(X_A-X_T)}\\
 &=  \sqrt{
(\left(
       \begin{array}{cc}
         2 & 2 \\
       \end{array}
     \right)-\left(
                \begin{array}{cc}
                  0 &  0 \\
                \end{array}
              \right))' \left(
               \begin{array}{cc}
                  1 &  .9 \\
                   .9 & 1 \\
               \end{array}
             \right)^{-1} (\left(
       \begin{array}{cc}
         2 & 2 \\
       \end{array}
     \right)-\left(
                \begin{array}{cc}
                    0 &  0 \\
                \end{array}
              \right))}\\
& =  \sqrt{
\left(
       \begin{array}{cc}
         2 &  2 \\
       \end{array}
     \right)' \left(
               \begin{array}{cc}
                   5.2 &  -4.7 \\
                   -4.7 & 5.2 \\
               \end{array}
             \right)\left(
       \begin{array}{cc}
         2 &  2 \\
       \end{array}
     \right)}\\
 &= 4.2\\
MD(X_B,X_{T}) &=  \sqrt{
\left(
       \begin{array}{cc}
         1.8 &  0 \\
       \end{array}
     \right)' \left(
               \begin{array}{cc}
                   5.2 &  -4.7 \\
                   -4.7 & 5.2 \\
               \end{array}
             \right)\left(
       \begin{array}{cc}
          1.8 &  0 \\
       \end{array}
     \right)}\\
 &= 17\\
\end{align*}
With $StataD(X_A,X_T)=\sqrt{(X_i-X_T)'diag(\Sigma_X^{-1})(X_i-X_T)}$ we find $StataD(X_A,X_T)=84$ and $StataD(X_B,X_T)=17$ since correlation is ignored.
\end{frame}

\begin{frame}{Mahalanobis Distance}
\vspace{-.2in}
\begin{figure}[ht] \centering
    \includegraphics[width = .9 \linewidth]{images/maha1.pdf}
\end{figure}

\end{frame}

\begin{frame}{Normalized Euclidean Distance}
\vspace{-.2in}
\begin{figure}[ht] \centering
    \includegraphics[width = .9 \linewidth]{images/maha2.pdf}
\end{figure}

\end{frame}

%\begin{frame}{Genmatch}
%
%\begin{figure}[ht] \centering
%    \includegraphics[width = .5 \linewidth]{images/genmatch_flow_chart.png}
%\end{figure}
%
%\end{frame}

%\begin{frame}{Genmatch}
%
%\begin{figure}[ht] \centering
%    \includegraphics[width = .8 \linewidth]{images/genmatch.png}
%\end{figure}
%
%\end{frame}

\begin{frame}{Local Methods and the Curse of Dimensionality}
\small
\textbf{Big} Problem: \pause in a mathematical space, the volume increases \alert{exponentially} when adding extra dimensions.
\vspace{-.5cm}
\begin{figure}[ht] \centering
    \includegraphics[width = .8 \linewidth]{images/curse_dimensionality1.pdf}
\end{figure}

\end{frame}

\begin{frame}{Local Methods and the Curse of Dimensionality}

\begin{figure}[ht] \centering
    \includegraphics[width = .6 \linewidth]{images/curse_dimensionality2.pdf}
\end{figure}

\end{frame}



\begin{frame}
\frametitle{The E-Harmony Problem}

Curse of dimensionality and on-line dating:\pause \\

\scalebox{0.75}{\includegraphics{eharmony.png}}
 
\invisible<1>{Suppose (for example) 29 dimensions are binary (0,1):} \pause \\
\invisible<1-2>{Suppose dimensions are independent:} \pause \\
\invisible<1-3>{Pr(2 people agree) = 0.5 } \pause 

\begin{eqnarray}
\invisible<1-4>{\text{Pr(Exact)} & = & \text{Pr(Agree)}_{1} \times \text{Pr(Agree)}_{2}\times \hdots \times \text{Pr(Agree)}_{29} } \pause  \nonumber \\
\invisible<1-5>{& = & 0.5 \times 0.5 \times \hdots \times 0.5} \pause  \nonumber \\
\invisible<1-6>{& = & 0.5^{29} } \pause \nonumber \\
\invisible<1-7>{& \approx & 1.8 \times 10^{-9}  } \pause    \nonumber 
\end{eqnarray}

\invisible<1-8>{1 in 536,870,912 people}  \pause 

\invisible<1-9>{Across many ``variables" (events) agreement is harder} 

\end{frame}




\begin{frame}{Matching with Bias Correction}

Matching estimators may behave badly if $X$ contains multiple continuous
variables.

Need to adjust matching estimators in the following way: \[
\tilde{\tau}_{ATT}= \frac{1}{N_1} \sum_{D_i=1}
(Y_i - Y_{j(i)}) - (\widehat\mu_0(X_i)-\widehat\mu_0(X_{j(i)})),
\] where $\mu_0(x)=E[Y|X=x, D=0]$ is the population regression function
under the control condition and $\widehat \mu_0$ is an estimate of
$\mu_0$.

$X_i-X_{j(i)}$ is often referred to as the \alert{matching discrepancy}.

These ``bias-corrected'' matching estimators behave well even if $\mu_0$
is estimated using a simple linear regression (ie.
$\mu_0(x)=\beta_0 + \beta_1 x$) (Abadie and Imbens, 2005)

\end{frame}

\begin{frame}{Matching with Bias Correction}

Each treated observation contributes \[
\mu_0(X_i)-\mu_0(X_{j(i)})
\] to the bias.

Bias-corrected matching: \[
\tilde\tau_{ATT}= \frac{1}{N_1}\sum_{D_i=1} \Big((Y_i-Y_{j(i)})-
(\widehat\mu_0(X_i)-\widehat\mu_0(X_{j(i)}))
\Big)
\] The large sample distribution of this estimator (for the case of
matching with replacement) is (basically) standard normal. $\mu_0$ is
usually estimated using a simple linear regression (ie.
$\mu_0(x)=\beta_0 + \beta_1 x$).\\\bigskip

In R: \texttt{Match(Y,Tr, X,BiasAdjust = TRUE)}

\end{frame}

\begin{frame}{Bias Adjustment with Matched Data}

\begin{tabular}{c|c|c|c|c}
 &Potential Outcome & Potential Outcome &   & \\
unit &under Treatment  & under Control &   &  \\

\hline
   i  & $Y_{1i}$  & $Y_{0i}$ &    $D_i$ &  $X_i$ \\
\hline
         1 & \multicolumn{ 1}{c|}{6} & \multicolumn{ 1}{c|}{\textcolor{red}{?}} &                    1 &          3 \\
         2 & \multicolumn{ 1}{c|}{1} & \multicolumn{ 1}{c|}{\textcolor{red}{?}} &                    1 &          1 \\
         3 & \multicolumn{ 1}{c|}{0} & \multicolumn{ 1}{c|}{\textcolor{red}{?}} &                    1 &          10 \\
\hline
         4 & \multicolumn{ 1}{c|}{} & \multicolumn{1}{c|}{0} &                    0 &          2 \\
         5 & \multicolumn{ 1}{c|}{} & \multicolumn{1}{c|}{9} &                    0 &          3 \\
         6 & \multicolumn{ 1}{c|}{} & \multicolumn{1}{c|}{1} &                    0 &          8 \\
         \hline
\end{tabular}

What is
$\tilde\tau_{ATT}= \frac{1}{N_1}\sum_{D_i=1} \Big((Y_i-Y_{j(i)})- (\widehat\mu_0(X_i)-\widehat\mu_0(X_{j(i)})) \Big)$?

\end{frame}

\begin{frame}{Bias Adjustment with Matched Data}

\footnotesize 

\begin{tabular}{c|c|c|c|c}
 &Potential Outcome & Potential Outcome &   & \\
unit &under Treatment  & under Control &   &  \\
\hline
   i  & $Y_{1i}$  & $Y_{0i}$ &    $D_i$ &  $X_i$ \\
\hline
         1 & \multicolumn{ 1}{c|}{6} & \multicolumn{ 1}{c|}{\textcolor{blue}{9}} &                    1 &          3 \\
         2 & \multicolumn{ 1}{c|}{1} & \multicolumn{ 1}{c|}{\textcolor{cyan}{0}} &                    1 &          1 \\
         3 & \multicolumn{ 1}{c|}{0} & \multicolumn{ 1}{c|}{\textcolor{green}{1}} &                    1 &          10 \\
\hline
         4 & \multicolumn{ 1}{c|}{} & \multicolumn{1}{c|}{0} &                    0 &          \textcolor{cyan}{2} \\
         5 & \multicolumn{ 1}{c|}{} & \multicolumn{1}{c|}{9} &                    0 &          \textcolor{blue}{3} \\
         6 & \multicolumn{ 1}{c|}{} & \multicolumn{1}{c|}{1} &                    0 &          \textcolor{green}{8} \\
         \hline
\end{tabular}

What is
$\tilde\tau_{ATT}= \frac{1}{N_1}\sum_{D_i=1} \Big((Y_i-Y_{j(i)})- (\widehat\mu_0(X_i)-\widehat\mu_0(X_{j(i)})) \Big)$?

Estimate $\widehat\mu_0(x)=\beta_0 + \beta_1 x = 5 - .4 x$. \pause Now
plug in:

\begin{eqnarray*}
\widehat{\tau}_{ATT} &=&1/3 \{ ((6-\textcolor{blue}{9})-(\widehat\mu_0(3)-\widehat\mu_0(\textcolor{blue}{3})))\\
&+&((1-\textcolor{cyan}{0})-(\widehat\mu_0(1)-\widehat\mu_0(\textcolor{cyan}{2})))\\
&+&((0-\textcolor{green}{1})-
(\widehat\mu_0(10)-\widehat\mu_0(\textcolor{green}{8})))\}\\
&=& -0.86
\end{eqnarray*}

Unadjusted:  $1/3((6-9) + (1-0) + (0-1) )=-1$

\end{frame}

\begin{frame}{Before Matching}
\vspace{-.3in}
\begin{figure}[ht] \centering
    \includegraphics[width = .75 \linewidth]{images/matching_sim1.pdf}
\end{figure}

\end{frame}

\begin{frame}{After Matching}
\vspace{-.3in}
\begin{figure}[ht] \centering
    \includegraphics[width = .75 \linewidth]{images/matching_sim2.pdf}
\end{figure}

\end{frame}

\begin{frame}{After Matching: Imputation Function}
\vspace{-.3in}
\begin{figure}[ht] \centering
    \includegraphics[width = .75 \linewidth]{images/matching_sim3.pdf}
\end{figure}

\end{frame}

\begin{frame}{After Matching: Imputation of missing $Y_0$}
\vspace{-.3in}
\begin{figure}[ht] \centering
    \includegraphics[width = .75 \linewidth]{images/matching_sim4.pdf}
\end{figure}

\end{frame}

\begin{frame}{After Matching: No Overlap in $Y_0$}
\vspace{-.3in}
\begin{figure}[ht] \centering
    \includegraphics[width = .75 \linewidth]{images/matching_sim5.pdf}
\end{figure}

\end{frame}

\begin{frame}{After Matching: Imputation of missing $Y_0$}
\vspace{-.3in}
\begin{figure}[ht] \centering
    \includegraphics[width = .75 \linewidth]{images/matching_sim6.pdf}
\end{figure}

\end{frame}

\begin{frame}{After Matching: Imputation of missing $Y_0$}
\vspace{-.3in}
\begin{figure}[ht] \centering
    \includegraphics[width = .75 \linewidth]{images/matching_sim7.pdf}
\end{figure}

\end{frame}

\begin{frame}{Choices when Matching}

\begin{itemize}
\itemsep1pt\parskip0pt\parsep0pt
\item
  With or Without Replacement? \pause
\item
  How many matches? \pause
\item
  Which Matching Algorithm?

  \begin{itemize}
  \itemsep1pt\parskip0pt\parsep0pt
  \item
    Genetic Matching
  \item
    Kernel Matching
  \item
    Full Matching
  \item
    Coarsened Exact Matching
  \item
    Matching as Pre-processing
  \item
    Propensity Score Matching \pause
  \end{itemize}
\item
  Use whatever gives you the best balance! Checking balance is important
  to get a sense for how much extrapolation is needed

  \begin{itemize}
  \itemsep1pt\parskip0pt\parsep0pt
  \item
    Should check balance on interactions and higher moments
  \end{itemize}
\item
  With insufficient overlap, all adjustment methods are problematic
  because we have to heavily rely on a model to impute missing potential
  outcomes.
\end{itemize}

\end{frame}

\begin{frame}{Balance Checks}

\begin{figure}[ht] \centering
    \includegraphics[width = .65 \linewidth]{images/hansen1.pdf}
\end{figure}

\end{frame}

\begin{frame}{Balance Checks}

\begin{figure}[ht] \centering
    \includegraphics[width = .99 \linewidth]{images/lyall1.pdf}
\end{figure}

\end{frame}

\begin{frame}{Variance of the Matching Estimator}

\begin{itemize}
\itemsep1pt\parskip0pt\parsep0pt
\item
  The sampling variance of the matching estimator (conditional on
  covariates and treatment indicators) can be written as the following:
  \[ V(\hat \tau_{ATT} | \mathbf{X}, \mathbf{D}) = \sum_{i=1}^{N} \lambda_i(\mathbf{X}, \mathbf{W}) \cdot \sigma^2_{D_i}(X_i)\]
  where $\lambda_i(\mathbf{X}, \mathbf{W})$ is just a set of weights which
  account for matching with replacement and multiple controls per
  treatment unit. \pause
\item
  $\sigma^2_i$ is the unit level variance. Abadie and Imbens suggest
  matching to estimate this quantity:

  \begin{itemize}
  \itemsep1pt\parskip0pt\parsep0pt
  \item
    Let $v(i)$ be the closest unit to $i$ with the same treatment
    indicator ($D_{v(i)} = D_i$). The sample variance of the outcome
    variable for these 2 units can be used to estimate
    $\sigma^2_{D_i}(X_i)$:
    \[ \hat \sigma^2_{D_i}(X_i) = (Y_i - Y_{v(i)})^2 / 2 \] 
    \item Robust standard errors also available\pause
  \end{itemize}
\item
  Do not use the bootstrap!
\end{itemize}

\end{frame}

\begin{frame}[fragile]{Useful Matching Functions}

The workhorse model is the \texttt{Match()} function in the
\texttt{Matching} package:

\small
\begin{verbatim}
Match(Y = NULL, Tr, X, Z = X, V = rep(1, length(Y)), 
    estimand = "ATT", M = 1, BiasAdjust = FALSE, exact = NULL, 
    caliper = NULL,  replace = TRUE, ties = TRUE, 
    CommonSupport = FALSE, Weight = 1, Weight.matrix = NULL, 
    weights = NULL, Var.calc = 0,  sample = FALSE, restrict = NULL, 
    match.out = NULL, distance.tolerance = 1e-05, 
    tolerance = sqrt(.Machine$double.eps), version = "standard") 
\end{verbatim}

Default distance metric (\texttt{Weight=1}) is normalized Euclidean distance

\begin{itemize}
\itemsep1pt\parskip0pt\parsep0pt
\item
  \texttt{MatchBalance(formu)} for balance checking
\item
  \texttt{GenMatch()} for genetic matching
\end{itemize}

\end{frame}

\subsection{Propensity Scores}\label{propensity-scores}

%\begin{frame}{Review: Law of Iterated Expectations}
%
%\begin{definition}
%A very important property of conditional expectations is:
%$$
%E[Y]=E[E[Y|X]]
%$$
%since $E[Y|X]$ is a function of $X$, the outer expectation
%is taken with respect to the distribution of $X$.
%\end{definition}
%
%Example: \pause 
%
%Imagine $Y$=Support and $X$=gender $(1,0)$. Then, the law of iterated
%expectations tells us: \[
%E[Y]=\pause E[Y|X=1]\cdot \Pr(X=1) + E[Y|X=0]\cdot \Pr(X=0).
%\] This means: \medskip
%\small
%
%\begin{tabular}{lcll}
%Average Support &=& Average Support for men&$\times$ Proportion of men\\
%               &+& Average Support for women&$\times$ Proportion of women.
%\end{tabular}
%
%\end{frame}

\begin{frame}{Identification with Propensity Scores}

\begin{definition}
Propensity score is defined as the selection probability conditional on the confounding variables:
$\pi(X)=\Pr(D=1|X)$
\end{definition}

\begin{iass}
\begin{enumerate}
\item $(Y_1,Y_0) \indep D |X$ (selection on observables)
\item $0<\Pr(D=1|X)<1$ with probability one (common support)
\end{enumerate}
\end{iass}

\begin{ires}
Under selection on observables we have $(Y_1,Y_0) \indep D| \pi(X)$, ie. conditioning
on the propensity score is enough to have independence between the treatment indicator and potential outcomes. Implies substantial dimension reduction.
\end{ires}

\end{frame}

\begin{frame}{Identification with Propensity Scores}

\begin{Proof}
\small
Show that $\Pr(D = 1|Y_0, Y_1, \pi(X)) = \Pr(D = 1|\pi(X)) = \pi(X)$, implying
independence of $(Y_0,Y_1)$ and $D$ conditional on $\pi(X)$.
\begin{eqnarray*}
\Pr(D=1|Y_1,Y_0,\pi(X))&=&E[D|Y_1,Y_0,\pi(X)]\\ \pause
&=&E\left[E[D|Y_1,Y_0,X]|Y_1,Y_0,\pi(X)\right] \mbox{ (LIE) }\\ \pause
&=&E\left[E[D|X]|Y_1,Y_0,\pi(X)\right] \mbox{ (SOO) } \\\pause
&=&E\left[\pi(X)|Y_1,Y_0,\pi(X)\right] \\ \pause
&=&\pi(X)
\end{eqnarray*}
Using a similar argument
\begin{eqnarray*}
\Pr(D=1|\pi(X))&=&E[D|\pi(X)]=E[E[D|X]|\pi(x)]\\ \pause
&=&E[\pi(X)|\pi(X)]=\pi(X)
\end{eqnarray*}
therefore $\Pr(D=1|Y_1,Y_0,\pi(X))=\Pr(D=1|\pi(X))$
\end{Proof}

\end{frame}

\begin{frame}{Matching on the Propensity Score}

\begin{corollary} If $(Y_1,Y_0) \indep D|X$, then
\begin{eqnarray*}
E[Y|D=1,\pi(X)=\bar{\pi}]-E[Y|D=0,\pi(X)=\bar{\pi}]&=&\\
E[Y_1-Y_0|\pi(X)=\bar{\pi}]
\end{eqnarray*}
\end{corollary}

Suggests a two step procedure to estimate causal effects under selection
on observables:

\begin{enumerate}
\def\labelenumi{\arabic{enumi}.}
\itemsep1pt\parskip0pt\parsep0pt
\item
  Estimate the propensity score $\pi(X)=P(D=1|X)$ (e.g. using
  logit/probit regression, machine learning methods, etc) \pause
\item
  Match or subclassify on propensity score.
\end{enumerate}

\end{frame}

\begin{frame}{Estimating the Propensity Score}

\begin{itemize}
\itemsep1pt\parskip0pt\parsep0pt
\item
  Given selection on observables we have $(Y_1,Y_0) \indep D|\pi(X)$
  which implies the balancing property of the propensity score: \[
    \Pr(X|D=1,\pi(X))=\Pr(X|D=0,\pi(X))
    \] \pause
\item
  We can use this to check if our estimated propensity score actually
  produces balance:
  $P(X|D=1,\widehat{\pi}(X))=P(X|D=0,\widehat{\pi}(X))$ \pause
\item
  To properly model the assignment mechanism, we need to include
  important confounders correlated with treatment and outcome
\item
  Need to find the correct functional form, miss-specified propensity
  scores can lead to bias. Any methods can be used (probit, logit, etc.)
\item
  Estimate $\mapsto$ Check Balance $\mapsto$ Re-estimate $\mapsto$ Check
  Balance
\end{itemize}

\end{frame}

\begin{frame}[fragile]{Example: Blattman (2010)}

\footnotesize

\begin{verbatim}
pscore.fmla <- as.formula(paste("abd~",paste(names(covar),collapse="+")))
abd <- data$abd
pscore_model <- glm(pscore.fmla, data = data,
family = binomial(link = logit))
pscore <- predict(pscore_model, type = "response")
\end{verbatim}
\centering
\includegraphics[height = 2.4in]{images/blattman_pscore.pdf}

\end{frame}

% \begin{frame}[fragile]{Blattman (2010): Match on the Propensity Score}

% \footnotesize

% \begin{verbatim}
% match.pscore <-Match(Tr=}\NormalTok{abd, }\DataTypeTok{X=}\NormalTok{pscore, }\DataTypeTok{M=}\DecValTok{1}\NormalTok{, }\DataTypeTok{estimand=}\StringTok{"ATT"}\NormalTok{)}
% \end{verbatim}

% \centering
% \includegraphics[height = 3in]{images/blattman_pscore_matching.pdf}

% \end{frame}

% \begin{frame}[fragile]{Blattman (2010): Check Balance}

% \footnotesize

% \begin{Shaded}
% \begin{Highlighting}[]
% \NormalTok{match.pscore <-}
% \NormalTok{+}\StringTok{ }\KeywordTok{MatchBalance}\NormalTok{(abd ~}\StringTok{ }\NormalTok{age, }\DataTypeTok{data=}\NormalTok{data, }\DataTypeTok{match.out =} \NormalTok{match.pscore)}

% \NormalTok{**}\ErrorTok{***}\StringTok{ }\NormalTok{(V1) age **}\ErrorTok{***}
% \StringTok{                       }\NormalTok{Before Matching     After Matching}
% \NormalTok{mean treatment........     }\FloatTok{21.366}          \FloatTok{21.366} 
% \NormalTok{mean control..........     }\FloatTok{20.151}          \FloatTok{20.515} 
% \NormalTok{std mean diff.........     }\FloatTok{24.242}          \FloatTok{16.976} 

% \NormalTok{var }\KeywordTok{ratio} \NormalTok{(Tr/Co).....     }\FloatTok{1.0428}         \FloatTok{0.98412} 
% \NormalTok{T-test p-value........  }\FloatTok{0.0012663}       \FloatTok{0.0034409} 
% \NormalTok{KS Bootstrap p-value..      }\FloatTok{0.016}           \FloatTok{0.034} 
% \NormalTok{KS Naive p-value......   }\FloatTok{0.024912}        \FloatTok{0.070191} 
% \NormalTok{KS Statistic..........    }\FloatTok{0.11227}        \FloatTok{0.077899} 
% \end{Highlighting}
% \end{Shaded}

% \end{frame}

% \begin{frame}[fragile]{Blattman (2010): Mahalanobis Distance Matchng}

% \footnotesize

% \begin{Shaded}
% \begin{Highlighting}[]
% \NormalTok{match.mah <-}\StringTok{ }\KeywordTok{Match}\NormalTok{(}\DataTypeTok{Tr=}\NormalTok{abd, }\DataTypeTok{X=}\NormalTok{covar, }\DataTypeTok{M=}\DecValTok{1}\NormalTok{, }\DataTypeTok{estimand=}\StringTok{"ATT"}\NormalTok{, }\DataTypeTok{Weight =} \DecValTok{3}\NormalTok{)}
% \KeywordTok{MatchBalance}\NormalTok{(abd ~}\StringTok{ }\NormalTok{age, }\DataTypeTok{data=}\NormalTok{data, }\DataTypeTok{match.out =} \NormalTok{match.mah)}

% \NormalTok{**}\ErrorTok{***}\StringTok{ }\NormalTok{(V1) age **}\ErrorTok{***}
% \StringTok{                       }\NormalTok{Before Matching     After Matching}
% \NormalTok{mean treatment........     }\FloatTok{21.366}          \FloatTok{21.366} 
% \NormalTok{mean control..........     }\FloatTok{20.151}          \FloatTok{21.154} 
% \NormalTok{std mean diff.........     }\FloatTok{24.242}          \FloatTok{4.2314} 

% \NormalTok{var }\KeywordTok{ratio} \NormalTok{(Tr/Co).....     }\FloatTok{1.0428}          \FloatTok{1.0336} 
% \NormalTok{T-test p-value........  }\FloatTok{0.0012663}      \FloatTok{3.0386e-05} 
% \NormalTok{KS Bootstrap p-value..      }\FloatTok{0.008}           \FloatTok{0.798} 
% \NormalTok{KS Naive p-value......   }\FloatTok{0.024912}         \FloatTok{0.94687} 
% \NormalTok{KS Statistic..........    }\FloatTok{0.11227}        \FloatTok{0.034261} 
% \end{Highlighting}
% \end{Shaded}

% \end{frame}

% \begin{frame}[fragile]{Blattman (2010): Genetic Matching}

% \footnotesize

% \begin{Shaded}
% \begin{Highlighting}[]
% \NormalTok{genout <-}\StringTok{ }\KeywordTok{GenMatch}\NormalTok{(}\DataTypeTok{Tr=}\NormalTok{abd,}\DataTypeTok{X=}\NormalTok{covar,}\DataTypeTok{BalanceMatrix=}\NormalTok{covar,}\DataTypeTok{estimand=}\StringTok{"ATT"}\NormalTok{,}
%                   \DataTypeTok{pop.size=}\DecValTok{1000}\NormalTok{)}
% \NormalTok{match.gen <-}\StringTok{ }\KeywordTok{Match}\NormalTok{(}\DataTypeTok{Tr=}\NormalTok{abd, }\DataTypeTok{X=}\NormalTok{covar,}\DataTypeTok{M=}\DecValTok{1}\NormalTok{,}\DataTypeTok{estimand=}\StringTok{"ATT"}\NormalTok{,}\DataTypeTok{Weight.matrix=}\NormalTok{genout)}
% \NormalTok{gen.bal <-}\StringTok{ }\KeywordTok{MatchBalance}\NormalTok{(abd~age,}\DataTypeTok{match.out=}\NormalTok{match.gen,}\DataTypeTok{data=}\NormalTok{covar)}

% \NormalTok{**}\ErrorTok{***}\StringTok{ }\NormalTok{(V1) age **}\ErrorTok{***}
% \StringTok{                       }\NormalTok{Before Matching     After Matching}
% \NormalTok{mean treatment........     }\FloatTok{21.366}          \FloatTok{21.366} 
% \NormalTok{mean control..........     }\FloatTok{20.151}          \FloatTok{21.225} 
% \NormalTok{std mean diff.........     }\FloatTok{24.242}          \FloatTok{2.8065} 

% \NormalTok{var }\KeywordTok{ratio} \NormalTok{(Tr/Co).....     }\FloatTok{1.0428}          \FloatTok{1.1337} 
% \NormalTok{T-test p-value........  }\FloatTok{0.0012663}         \FloatTok{0.21628} 
% \NormalTok{KS Bootstrap p-value..      }\FloatTok{0.008}           \FloatTok{0.454} 
% \NormalTok{KS Naive p-value......   }\FloatTok{0.024912}         \FloatTok{0.68567} 
% \NormalTok{KS Statistic..........    }\FloatTok{0.11227}        \FloatTok{0.046512} 
% \end{Highlighting}
% \end{Shaded}

% \end{frame}

\begin{frame}{Weighting on the Propensity Score}

\small
Provided that the relevant moments exists, if $Y_1,Y_0\indep D |X$, then

\begin{eqnarray*}
\tau_{ATE}&=&E[Y_1-Y_0]=E\left[
Y\cdot \frac{D-\pi(X)}{\pi(X)\cdot (1-\pi(X))}
\right]\\
\tau_{ATT}&=&E[Y_1-Y_0|D=1]=\frac{1}{P(D=1)}\cdot E\left[
Y\cdot \frac{D-\pi(X)}{1-\pi(X)}
\right]
\end{eqnarray*}

\begin{Proof}
\scriptsize Consider
\begin{eqnarray*}
&&E\left[\left.
Y \frac{D-\pi(X)}{\pi(X) (1-\pi(X))}\right| X \right]\\
&=& E\left[\left.\frac{Y}{\pi(X)} \right|X, D=1 \right] \pi(X) + E\left[\left.\frac{-Y}{1-\pi(X)}\right|X, D=0\right] (1-\pi(X))\\
&=& E[Y|X, D=1] - E[Y|X, D=0]
\end{eqnarray*}
And the results of the proposition follow from integration over $\Pr(X)$ and $\Pr(X|D=1)$.
\end{Proof}

\end{frame}

\begin{frame}{Weighting on the Propensity Score}

\small

\begin{eqnarray*}
\tau_{ATE}&=&E[Y_1-Y_0]=E\left[
Y\cdot \frac{D-\pi(X)}{\pi(X)\cdot (1-\pi(X))}
\right]\\
\tau_{ATT}&=&E[Y_1-Y_0|D=1]=\frac{1}{\Pr(D=1)}\cdot E\left[
Y\cdot \frac{D-\pi(X)}{1-\pi(X)}
\right]
\end{eqnarray*}

How do we estimate this? \pause Analogy principle suggest a two step
procedure:

\begin{enumerate}
\def\labelenumi{\arabic{enumi}.}
\itemsep1pt\parskip0pt\parsep0pt
\item
  Estimate the propensity score ($\widehat{\pi}(X)$)
\item
  Use sample averages and the estimated propensity score to produce
  analog estimators of $\tau_{ATE}$ and $\tau_{ATT}$:

  \begin{eqnarray*}
  \widehat{\tau}_{ATE}&=&\frac{1}{N} \sum_{i=1}^N
  Y_i\cdot \frac{D_i-\widehat{\pi}(X_i)}{\widehat{\pi}(X_i)\cdot
  (1-\widehat{\pi}(X_i))},\\
  \widehat{\tau}_{ATT}&=&\frac{1}{N_1} \sum_{i=1}^N
  Y_i\cdot \frac{D_i-\widehat{\pi}(X_i)}{1-\widehat{\pi}(X_i)}
  \end{eqnarray*}
\end{enumerate}

\end{frame}

% \begin{frame}[fragile]{Blattman (2010): Weighting on the Propensity
% Score}

% \begin{Shaded}
% \begin{Highlighting}[]
% \KeywordTok{mean}\NormalTok{(covar$age[abd ==}\StringTok{ }\DecValTok{1}\NormalTok{]) -}\StringTok{ }\KeywordTok{mean}\NormalTok{(covar$age[abd ==}\StringTok{ }\DecValTok{0}\NormalTok{])}
% \NormalTok{[}\DecValTok{1}\NormalTok{] }\FloatTok{1.215263}

% \KeywordTok{sum}\NormalTok{(covar$age *}\StringTok{ }\NormalTok{(abd -}\StringTok{ }\NormalTok{pscore) /}\StringTok{ }\NormalTok{(}\DecValTok{1} \NormalTok{-}\StringTok{ }\NormalTok{pscore)) /}\StringTok{ }\KeywordTok{length}\NormalTok{(abd)}
% \NormalTok{[}\DecValTok{1}\NormalTok{] }\FloatTok{0.007663878}
% \end{Highlighting}
% \end{Shaded}

% \end{frame}

%\begin{frame}{Caveats about Inverse Propensity Score Weighting}
%
%\begin{itemize}
%\itemsep1pt\parskip0pt\parsep0pt
%\item
%  Shown to have poor finite sample performance when some weights are
%  ``extreme'', which indicates a lack of overlap (Kang and Schaefer
%  2008).
%\item
%  Might want to trim units from the sample with extreme weights, though
%  this will change your estimand. See Crump, Hotz, Imbens, and Mitnik
%  (2009) for more on this. \pause
%\item
%  \textbf{Entropy Balancing}: To avoid the iterative process of PS
%  weighting and matching, choose weights that optimize for balance
%  directly (Hainmueller 2011).
%\end{itemize}
%
%\end{frame}

\subsection{Regression}\label{regression}

%\begin{frame}{Regression}
%
%David Freedman:
%
%\begin{quote}
%I sometimes have a nightmare about Kepler. Suppose a few of us were
%transported back in time to the year 1600, and were invited by the
%Emperor Rudolph II to set up an Imperial Department of Statistics in the
%court at Prague. Despairing of those circular orbits, Kepler enrolls in
%our department. We teach him the general linear model, least squares,
%dummy variables, everything. He goes back to work, fits the best
%circular orbit for Mars by least squares, puts in a dummy variable for
%the exceptional observation - and publishes. And that's the end, right
%there in Prague at the beginning of the 17th century.
%\end{quote}
%
%\end{frame}

%\begin{frame}{Linear Regression with Potential Outcomes (No $X$s)}
%
%\begin{eqnarray*}
%Y_i &=& D_i Y_{1i} + (1-D_i)Y_{0i} \\ \pause
%    &=& Y_{0i} + (Y_{1i}- Y_{0i})D_i \\ \pause
%    &=& Y_{0i} + \tau_i D_i \\ \pause
%    &=& \mu_0 + \tau_i D_i + \nu_0
%\end{eqnarray*}
%
%where $\mu_0 = E[Y_0]$ and $\nu_0=Y_{0i}-E[Y_0]$. \pause To identify
%$\tau_{ATE}$ we re-express: \[
%Y_i = \mu_0 + (\mu_1 - \mu_0) D_i + Vepsilon_i
%\] where $\mu_1 = E[Y_1]$, $\nu_1=Y_{1i}-E[Y_1]$, and so
%$(\mu_1 - \mu_0)=\tau_{ATE}$. The regression error term is
%$Vepsilon_i = \nu_0 + D_i(\nu_1 - \nu_0)$. \pause Note that:
%
%\[
%Vepsilon_i = \left\{
% \begin{array}{l}
%  \nu_0 = Y_{0i}-E[Y_0]\, \mbox{ if }\, D_i=0\\
%  \nu_1 = Y_{1i}-E[Y_1]\, \mbox{ if }\, D_i=1
% \end{array}
% \right\}
%\] When will $Vepsilon_i \indep D_i$ so that $\widehat{\tau}_{ATE}$
%is unbiased and consistent?
%
%\end{frame}
%
%\begin{frame}{Linear Regression with Potential Outcomes (No $X$s)}
%
%\[
%Y_i = \mu_0 + (\mu_1 - \mu_0) D_i + Vepsilon_i
%\] with $Vepsilon_i = \nu_0 + D_i(\nu_1 - \nu_0)$ and so \[
%Vepsilon_i = \left\{
% \begin{array}{l}
%  \nu_0 = Y_{0i}-E[Y_0]\, \mbox{ if }\, D_i=0\\
%  \nu_1 = Y_{1i}-E[Y_1]\, \mbox{ if }\, D_i=1
% \end{array}
% \right.
%\] $Vepsilon_i$ and $D_i$ can be correlated in two ways:\pause
%
%\begin{enumerate}
%\def\labelenumi{\arabic{enumi}.}
%\itemsep1pt\parskip0pt\parsep0pt
%\item
%  If there is a baseline bias $E[Y_0|D=1]\neq E[Y_0|D=0]$ then $D_i$ is
%  correlated with $\nu_0$ and thus correlated with $Vepsilon_i$.
%
%  \begin{itemize}
%  \itemsep1pt\parskip0pt\parsep0pt
%  \item
%    This is true even if $\tau_i=Y_{1i}-Y_{0i}$ is constant for all
%    $i$, so $D(\nu_1 - \nu_0)=0$ on average because $(\nu_1 - \nu_0)$
%    does not vary with $D$.\pause
%  \end{itemize}
%\item
%  If the magnitude of the treatment effects $\tau_i$ is correlated
%  with membership in the treatment group, then $(\nu_1 - \nu_0)$ varies
%  for the treated and untreated so $Vepsilon_i$ is correlated with
%  $D_i$ through $D(\nu_1 - \nu_0)$.
%
%  \begin{itemize}
%  \itemsep1pt\parskip0pt\parsep0pt
%  \item
%    This holds even with no baseline bias $E[Y_0|D=1]=E[Y_0|D=0]$ when
%    $D$ is uncorrelated with $\nu_0$
%  \end{itemize}
%\end{enumerate}
%
%\end{frame}

\begin{frame}{Identification under Selection on Observables: Regression}

Consider the linear regression of
$Y_i = \beta_0 + \tau D_i  + X_i' \beta + \epsilon_i$.\\

Given \alert{selection on observables}, there are mainly three identification
scenarios: \pause

\begin{enumerate}
\def\labelenumi{\arabic{enumi}.}
\itemsep1pt\parskip0pt\parsep0pt
\item
  Constant treatment effects and outcomes are linear in $X$

  \begin{itemize}
  \itemsep1pt\parskip0pt\parsep0pt
  \item
    $\tau$ will provide unbiased and consistent estimates of ATE.
    \bigskip
  \end{itemize}
\item
  Constant treatment effects and unknown functional form

  \begin{itemize}
  \itemsep1pt\parskip0pt\parsep0pt
  \item
    $\tau$ will provide well-defined linear approximation to the
    average causal response function $E[Y|D=1,X]-E[Y|D=0,X]$.
    Approximation may be very poor if $E[Y|D,X]$ is misspecified and
    then $\tau$ may be biased for the ATE. \pause
  \end{itemize}
\item
  Heterogeneous treatment effects ($\tau$ differs for different values
  of $X$)

  \begin{itemize}
  \itemsep1pt\parskip0pt\parsep0pt
  \item
    If outcomes are linear in $X$, $\tau$ is unbiased and consistent
    estimator for conditional-variance-weighted average of the
    underlying causal effects. This average is often different from the
    ATE.
  \end{itemize}
\end{enumerate}

\end{frame}

\begin{frame}{Identification under Selection on Observables: Regression}

\begin{iass}

\begin{enumerate}

\item Constant treatment effect: $\tau = Y_{1i}-Y_{0i}$ for all $i$
\item Control outcome is linear in $X$: $Y_{0i} = \beta_0 + X_i' \beta + \epsilon_i$ with $\epsilon_i \indep X_i$ (no omitted variables and linearly separable confounding)

\end{enumerate}

\end{iass}

\begin{ires}

Then $\tau_{ATE}= E[Y_1-Y_0]$ is identified by a regression of the observed outcome on the covariates and the treatment indicator  $Y_i= \beta_0 + \tau D_i + X_i' \beta + \epsilon_i$

\end{ires}

\end{frame}

%\begin{frame}{Linear Constant Effects Model}
%
%Assume \alert{constant linear effects} and
%\alert{linearly separable confounding}:
%\[Y_i(d) = Y_i = \beta_0 + \tau D_i + \eta_i\] \pause
%
%\begin{itemize}
%\itemsep1pt\parskip0pt\parsep0pt
%\item
%  \textbf{Linearly separable confounding:} assume that
%  $E[\eta_i |     X_i] = X'_i \beta$, which means that
%  $\eta_i = X'_i \beta +     \epsilon_i$ where
%  $E[\epsilon_i | X_i] = 0$.
%\item
%  Under this model, $(Y_1,Y_0) \indep D |X$ implies
%  $\epsilon_i | X \indep D$
%\item
%  As a result,
%
%  \begin{align*}
%       Y_i &= \beta_0 + \tau D_i + E[\eta_i] \\
%      &= \beta_0 + \tau D_i + X'_i \beta +
%      E[\epsilon_i] \\
%      &= \beta_0 + \tau D_i + X'_i \beta 
%     \end{align*}
%\item
%  Thus, a regression where $D_i$ and $X_i$ are entered linearly can
%  recover the ATE.
%\end{itemize}
%
%\end{frame}
%
%\begin{frame}{Omitted Variable Bias}
%
%\small
%
%What happens if we ignore or do not measure $X_i$? Say we have the
%following \textbf{long} model:
%
%\[ Y_i = \beta_0 + \tau D_i + \beta_1 X_{1i} + \beta_2 X_{2i} + \epsilon_i\]
%
%\pause
%
%If we do not observe or measure $X_{2i}$ then we put it in the error
%term and write the \textbf{short} equation:
%
%\[ Y_i = \beta_0 + \tau D_i + \beta_1 X_{1i} + \eta_i\] where
%$\eta_i = \beta_2 X_{2i} + \epsilon_i$
%
%Write the regression of $X_{2i}$ onto the explanatory variables as:
%\[X_{2i} = \delta_0 + \delta_{D} D_i + \delta_1 X_{1i} + r_i\]
%
%\pause
%
%Plugging into the long equation:
%\[Y_i = (\beta_0 + \beta_2 \delta_0) + (\tau + \beta_2 \delta_{D})D_i +(\beta_1 + \beta_2 \delta_1) X_{1i} + \epsilon_i + \beta_2 r_i\]
%
%\pause
%
%Thus, the \textbf{Omitted Variable Bias} formula is:
%\[\hat \tau = \tau + \beta_2 \cdot \delta_D\]
%
%\end{frame}

%\begin{frame}{Regression with Heterogeneous Effects}
%
%What is regression estimating when we allow for heterogeneity? \pause
%
%First, let's re-derive the ATT matching estimator so we can compare it
%to the regression estimator:
%
%Assuming $(Y_1,Y_0) \indep D | X$ and let $\tau_x$ be the conditional
%ATE $\tau_x = E[Y_1 - Y_0 | X = x]$ (note that X is discrete):
%
%\begin{align*}
%\tau_{ATT} &= E[Y_1 - Y_0 | D = 1] \\
%&= \E\{ E[Y_1 - Y_1 | X, D = 1] | D = 1 \}\\
%&= \E\{ E[Y_1| X, D = 1] - E[Y_0| X, D = 1] | D = 1 \}  \\
%&= \E\{ E[Y | X, D = 1] - E[Y | X, D = 0] | D = 1 \} \text{ (by CIA)} \\
%&= E[\tau_x | D = 1] = \sum_x \tau_x \Pr[X = x | D = 1] \\ 
%&= \frac{\sum_x \tau_x \Pr[D = 1 | X = x] \Pr[X=x]}{\sum_x \Pr[D = 1 | X = x] \Pr[X=x]}
%\end{align*}
%
%\end{frame}

\begin{frame}{Regression with Heterogeneous Effects}

What is regression estimating when we allow for heterogeneity? \pause


Suppose that we wanted to estimate $\tau_{OLS}$ using a
\alert{fully saturated} regression model:
\[ Y_i = \sum_x B_{xi} \beta_x + \tau_{OLS} D_i + e_i\] where $B_{xi}$
is a dummy variable for unique combination of $X_i$.

Because this regression is fully saturated, it is linear in the
covariates (i.e.~linearity assumption holds by construction). %\pause

%We will also use the following property of OLS (Frisch-Waugh-Lovell
%(FWL) theorem):
%\[\beta_k = \frac{\Cov(Y_i, \tilde{x}_{ki})}{V(\tilde{x}_{ki})}\]
%where $\tilde{x}_{ki}$ is the regression of $x_{ki}$ on all other
%covariates.

\end{frame}
%
%\begin{frame}{Regression with Heterogeneous Effects}
%
%What is $\tau_{OLS}$? Using FWL,
%
%\begin{align*}
% \tau_{OLS} &= \frac{\Cov(Y, \tilde D)}{V(\tilde D)} \\
% &= \frac{\Cov(E[Y|X,D], D - E[D|X])}{\E(D - E[D|X])^2} \\
%  &= \frac{\E\{E[Y|X,D] (D - E[D|X])\}}{\E(D - E[D|X])^2} \\
% &= \frac{\E\{(E[Y|X,D = 0] + \tau_x D) (D - E[D|X])\}}{\E(D - E[D|X])^2}\\
% &= \frac{E[\tau_x (D - E[D|X])^2]}{E[(D -  E[D|X]^2)]} = \frac{E[\tau_x  E[(D - E[D|X])^2|X]]}{E[E[(D -  E[D|X]^2)|X]]} = \frac{E[\tau_x \sigma^2_{D,X}]}{E[\sigma^2_{D,X}]} \\
% &= \frac{\sum_x \tau_x [\Pr[D = 1 | X = x] (1 - \Pr[D = 1 | X = x])] \Pr[X=x]}{
%   [\Pr[D = 1 | X = x] (1 - \Pr[D = 1 | X = x])] \Pr[X=x]}
%\end{align*}
%
%\end{frame}

\begin{frame}{Heterogenous Treatment Effects}

\scriptsize

With two $X$ strata there are two stratum-specific average causal
effects that are averaged to obtain the ATE or ATT.

Subclassification weights the stratum-effects by the marginal
distribution of $X$, i.e. weights are proportional to the share of units in each stratum:

\begin{eqnarray*}
% \nonumber to remove numbering (before each equation)
\tau_{ATE} &=& (E[Y|D=1,X=1]-E[Y|D=0,X=1])\textcolor{blue}{\Pr(X=1)} \\
   &+&   (E[Y|D=1,X=2]-E[Y|D=0,X=2])\textcolor{blue}{\Pr(X=2)}
\end{eqnarray*}

Regression weights by the marginal distribution of $X$ \alert{and} the
conditional variance of $V[D|X]$ in each stratum:

\begin{eqnarray*}
% \nonumber to remove numbering (before each equation)
  \tau_{OLS} &=& (E[Y|D=1,X=1]- E[Y|D=0,X=1])\frac{\textcolor{cyan}{V[D|X=1]}\textcolor{blue}{\Pr(X=1)}}{\sum_{X}Var[D|X=x]\Pr(X=x)} \\
   &+& (E[Y|D=1,X=2]- E[Y|D=0,X=2])\frac{\textcolor{cyan}{V[D|X=2]}\textcolor{blue}{\Pr(X=2)}}{\sum_{X}V[D|X=x]\Pr(X=x)}
\end{eqnarray*}

\begin{itemize}
\itemsep1pt\parskip0pt\parsep0pt
\item
  So strata with a higher $V[D|X]$ receive higher weight. These are
  the strata with propensity scores close to $.5$
\item
  Strata with propensity score close to $0$ or $1$ receive lower weight
\item
  OLS is a minimum-variance estimator of $\tau_{OLS}$ so it
  downweights strata where the average causal effects are less precisely
  estimated.
\end{itemize}

\end{frame}

\begin{frame}{Heterogenous Treatment Effects}

\scriptsize

With two $X$ strata there are two stratum-specific average causal
effects that are averaged to obtain the ATE or ATT.

Subclassification weights the stratum-effects by the marginal
distribution of $X$, i.e. weights are proportional to the share of units in each stratum:


\begin{eqnarray*}
% \nonumber to remove numbering (before each equation)
\tau_{ATE} &=& (E[Y|D=1,X=1]-E[Y|D=0,X=1])\textcolor{blue}{\Pr(X=1)} \\
   &+&   (E[Y|D=1,X=2]-E[Y|D=0,X=2])\textcolor{blue}{\Pr(X=2)}
\end{eqnarray*}

Regression weights by the marginal distribution of $X$ \alert{and} the
conditional variance of $V[D|X]$ in each stratum:

\begin{eqnarray*}
% \nonumber to remove numbering (before each equation)
  \tau_{OLS} &=& (E[Y|D=1,X=1]- E[Y|D=0,X=1])\frac{\textcolor{cyan}{V[D|X=1]}\textcolor{blue}{\Pr(X=1)}}{\sum_{X}Var[D|X=x]\Pr(X=x)} \\
   &+& (E[Y|D=1,X=2]- E[Y|D=0,X=2])\frac{\textcolor{cyan}{V[D|X=2]}\textcolor{blue}{\Pr(X=2)}}{\sum_{X}V[D|X=x]\Pr(X=x)}
\end{eqnarray*}

\begin{itemize}
\itemsep1pt\parskip0pt\parsep0pt
\item
  Whenever both weighting components are misaligned (e.g.~the PS is
  close to 0 or 1 for relatively large strata) then $\tau_{OLS}$
  diverges from $\tau_{ATE}$ or $\tau_{ATT}$.
\item
  With constant effects we have $\tau_{OLS}=\tau_{ATE}=\tau_{ATT}$
\item
  If linearity fails the intuition remains that linearity in $X$ implies
  an implicit linearity in the underlying PS.
\end{itemize}

\end{frame}

\begin{frame}{Conclusion: Regression}
\small
\begin{center}
\fbox{\alert{Is regression evil? \frownie}}
\end{center}

\pause

\begin{itemize}
\itemsep1pt\parskip0pt\parsep0pt
\item
  Its ease sometimes results in lack of thinking. So only a little.
  \smiley
\item
  For descriptive inference, very useful!

  \begin{itemize}
  \itemsep1pt\parskip0pt\parsep0pt
  \item
    Good tool for characterizing the conditional expectation function
    (CEF)
  \item
    But other less restrictive tools are also available for that task
    (machine learning) \pause
  \end{itemize}
\item
  For causal analysis, always need to ask yourself if \emph{linearly}
  separable confounding is plausible.

  \begin{itemize}
  \itemsep1pt\parskip0pt\parsep0pt
  \item
    A regression is causal when the CEF it approximates is causal.\\
  \item
    Still need to check common support! 
    \item Results will be highly sensitive if the treated and controls are far apart (e.g. standardized difference above .2)
  \end{itemize}
\item
  Think about what your \textbf{estimand} is: because of variance
  weighting, coefficient from your regression may not capture ATE if effects are heterogeneous
\end{itemize}

\end{frame}

\section{When Do Observational Studies Recover Experimental
Benchmarks?}\label{when-do-observational-studies-recover-experimental-benchmarks}

\begin{frame}
  \frametitle{Dehejia and Wabha (1999) Results}
  \includegraphics[height=2.7in,keepaspectratio=1]{images/tableDW.pdf}
\end{frame}

\begin{frame}{Heckman, Ichimura, Smith, and Todd (1998)}

\begin{itemize}
\itemsep1pt\parskip0pt\parsep0pt
\item
  Study randomized evaluation of the Job Training Partnership Act (JTPA)
\item
  3 groups:

  \begin{itemize}
  \itemsep1pt\parskip0pt\parsep0pt
  \item
    Experimental Treatment Group
  \item
    Experimental Control Group
  \item
    Eligible Non-Participants (ENPs): eligible for program but chose not
    to participate
  \end{itemize}
\item
  Heckman et al. studies under what conditions can covariate adjustment
  methods make the differences between ENPs and experimental controls
  diappear.
\end{itemize}

\end{frame}

\begin{frame}{Heckman, Ichimura, Smith, and Todd (1998)}

\begin{figure}[ht] \centering
    \includegraphics[width = .8 \linewidth]{images/heckman_hist.pdf}
\end{figure}

\end{frame}

\begin{frame}{Heckman, Ichimura, Smith, and Todd (1998)}

Decomposes bias: \[ B = B_1 + B_2 + B_3 \]

where

\begin{itemize}
\itemsep1pt\parskip0pt\parsep0pt
\item
  $B_1$ is bias due to lack of common support \pause
\item
  $B_2$ is bias from the fact that distribution of outcome ($Y_0$) is
  different among ENPs and experimental controls \pause
\item
  $B_3$ is differences in outcomes that remain even after controlling
  for observed differences (\alert{hidden bias}).
\end{itemize}

\end{frame}

\begin{frame}{Characterizing Selection Bias}

\begin{figure}[ht] \centering
    \includegraphics[width = .95 \linewidth]{images/heckman_table.pdf}
\end{figure}

\end{frame}

\begin{frame}{Characterizing Selection Bias}

\begin{itemize}
\item
  Matching removes most of the bias, but because experimental estimates
  are small, point estimates would still be substantially biased.
  \frownie \pause
\item
  Finds that the bias is roughly constant over time. \pause
\item
  Thus, difference-in-differences estimator (combined with matching)
  drives the bias to near 0. \smiley 

  \begin{itemize}
  \itemsep1pt\parskip0pt\parsep0pt
  \item
    More about difference-in-differences estimator later in the
quarter
  \end{itemize}
\end{itemize}

\end{frame}

%\begin{frame}{Shadish, Clark, and Steiner (2008)}
%
%\begin{figure}[ht] \centering
%    \includegraphics[width = .95 \linewidth]{images/shadish_image.png}
%\end{figure}
%
%\end{frame}
%
%\begin{frame}{Shadish, Clark, and Steiner (2008)}
%
%\begin{figure}[ht] \centering
%    \includegraphics[width = 1 \linewidth]{images/shadish_table.png}
%\end{figure}
%
%\end{frame}

%\section{Placebo Tests}\label{placebo-tests}
%
%\begin{frame}{Post-Mortem Atlantic Salmon}
%
%\begin{figure}[ht] \centering
%    \includegraphics[width = 1 \linewidth]{images/salmon.png}
%\end{figure}
%
%\end{frame}
%
%\begin{frame}{Tests of Design}
%
%\begin{itemize}
%\itemsep1pt\parskip0pt\parsep0pt
%\item
%  Imagine we have data on a ``placebo'' outcome that is known to be
%  unaffected by the treatment
%
%  \begin{itemize}
%  \itemsep1pt\parskip0pt\parsep0pt
%  \item
%    E.g. lags of the outcome variable that are measured before treatment
%  \end{itemize}
%\item
%  For example, assume $D$ is realized at $t=0$ and is ignorable
%  conditional on a set of $T$ lags of the outcome
%  $Y_1,Y_0\indep D | Y_{t=-1},Y_{t=-2},...,Y_{t=-T},X$\medskip
%\item
%  Given a stability assumption we should have ignorability conditional
%  on all lags but one:\[
%  Y_{t=-1} \indep D | Y_{t=-2},...,Y_{t=-T},X
%  \]
%\item
%  If we find a non-zero placebo effect for the first lag, then
%  ignorability conditional on all lags seems not very credible (esp.
%  with many lags).
%\end{itemize}
%
%\end{frame}
%
%\begin{frame}{Sekhon and Titiunik}
%
%\begin{figure}[ht] \centering
%    \includegraphics[width = 1 \linewidth]{images/sekhon_new_neighbors.png}
%\end{figure}
%
%\end{frame}
%
%\begin{frame}{Sekhon and Titiunik}
%
%\begin{figure}[ht] \centering
%    \includegraphics[width = 1 \linewidth]{images/sekhon_texas.png}
%\end{figure}
%
%\end{frame}
%
%\begin{frame}{Sekhon and Titiunik: Failed Placebo Test}
%
%\begin{figure}[ht] \centering
%    \includegraphics[width = 1 \linewidth]{images/sekhon_failed_placebo.png}
%\end{figure}
%
%\end{frame}
%
%\begin{frame}{Sekhon and Titiunik: Placebo Test - Incumbent Vote (2002)}
%
%\begin{table}[tb]
%  \begin{center}
%   \caption{Placebo Estimates Under Two Different Designs}
%   \begin{tabular}{llll}
%   \hline
% \hline
% \textbf{Design} & \textbf{Estimate} & \textbf{95\% CI} & \textbf{p value}\\
% \hline
%   Presidential Vote Only & 0.0285 & [0.0160, 0.0413] & 0.00 \\
%   Full Set of Controls & 0.00245 & [-.00488, 0.00954] & 0.513\\
% \hline
% \hline
% \end{tabular}
%\end{center}
%\end{table}
%
%\end{frame}
%
%\begin{frame}{Estimates on Experimental JTPA Data}
%
%\begin{table}
%\begin{center}
%\begin{tabular}{l|ccc}
%           & \multicolumn{ 3}{c}{Earnings75 Outcome} \\
%           &       mean &         se &     t-stat \\
%\hline
% Simple Difference &       0.27 &        0.30 &        0.9 \\
%OLS (parallel) &       0.15 &       0.22 &        0.7 \\
%OLS (separate) &       0.12 &       0.22 &        0.6 \\
%Propensity Score Weighting &       0.15 &        0.30 &        0.5 \\
%Propensity Score Blocking &        0.10 &       0.03 &        3.4 \\
%Propensity Score Regression &       0.16 &        0.30 &        0.5 \\
%Propensity Score Matching &       0.23 &       0.37 &        0.6 \\
%  Matching &       0.14 &       0.28 &        0.50 \\
%Weighting and Regression &       0.15 &       0.21 &        0.7 \\
%Blocking and Regression &       0.09 &       0.02 &        3.8 \\
%Matching and Regression &       0.06 &       0.28 &        0.2 \\
%\end{tabular}
%\end{center}
%\end{table}
%
%\end{frame}
%
%\begin{frame}{Estimats on CPS Data}
%
%\begin{table}
%\begin{center}
%\begin{tabular}{l|ccc}
%           & \multicolumn{ 3}{c}{Earnings75 Outcome} \\
%           &       mean &         se &     t-stat \\
%\hline
%Simple Difference &        -12 &       0.68 &        -18 \\
%OLS (parallel) &       -1.2 &       0.36 &         -3 \\
%OLS (separate) &       -1.1 &       0.36 &         -3 \\
%Propensity Score Weighting &       -1.2 &       0.26 &         -5 \\
%Propensity Score Blocking &       -1.4 &       0.32 &         -9 \\
%Propensity Score Regression &       -1.7 &       0.79 &         -2 \\
%Propensity Score Matching &       -1.3 &       0.46 &         -3 \\
%  Matching &       -1.3 &       0.41 &         -3 \\
%Weighting and Regression &       -1.2 &       0.24 &         -5 \\
%Blocking and Regression &       -1.3 &       0.25 &         -5 \\
%Matching and Regression &       -1.3 &       0.42 &         -3 \\
%\end{tabular}
%\end{center}
%\end{table}
%
%\end{frame}
%
%\begin{frame}{Placebo Outcome with Zero Effect}
%
%\begin{itemize}
%\itemsep1pt\parskip0pt\parsep0pt
%\item
%  Similar placebo tests have been successfully used for outcomes that
%  are known to be unaffected by the treatment:
%
%  \begin{itemize}
%  \itemsep1pt\parskip0pt\parsep0pt
%  \item
%    Several studies support Becker and Murphy's (1988) theory of
%    rational addiction for tobacco and alcohol consumption. Auld and
%    Grootendorst (2004, JHE) replicate the exact same models with data
%    for milk, eggs, oranges, and apples. \pause
%  \item
%    Krueger (1993) reports that the ability to use computers causes a
%    15-20\% increase in earnings via a regression analysis of
%    cross-sectional data. Using a similar design, Dinardo and
%    Pischke,report that the use of calculators, telephones, pens or
%    pencils, and chairs while on the job ``cause'' a nearly equivalent
%    increase in wages. \pause
%  \item
%    Several studies have found significant networks effects on outcomes
%    such as obesity, smoking, alcohol use, and happiness. Cohen-Cole and
%    Fletcher (2008, BMJ) use similar models and data and find similar
%    network effects for acne, height, and headaches.
%  \end{itemize}
%\end{itemize}
%
%\end{frame}

\end{document}
