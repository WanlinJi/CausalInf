\documentclass{beamer}

%\usepackage[table]{xcolor}
\mode<presentation> {
  \usetheme{Boadilla}
%  \usetheme{Pittsburgh}
%\usefonttheme[2]{sans}
\renewcommand{\familydefault}{cmss}
%\usepackage{lmodern}
%\usepackage[T1]{fontenc}
%\usepackage{palatino}
%\usepackage{cmbright}
  \setbeamercovered{transparent}
\useinnertheme{rectangles}
}
%\usepackage{normalem}{ulem}
%\usepackage{colortbl, textcomp}
\setbeamercolor{normal text}{fg=black}
\setbeamercolor{structure}{fg= black}
\definecolor{trial}{cmyk}{1,0,0, 0}
\definecolor{trial2}{cmyk}{0.00,0,1, 0}
\definecolor{darkgreen}{rgb}{0,.4, 0.1}
\usepackage{array}
\beamertemplatesolidbackgroundcolor{white}  \setbeamercolor{alerted
text}{fg=red}
\newtheorem{assumption}{Assumption}

\setbeamertemplate{caption}[numbered]\newcounter{mylastframe}

%\usepackage{color}
\usepackage{tikz}
\usetikzlibrary{arrows}
\usepackage{colortbl}
%\usepackage[usenames, dvipsnames]{color}
%\setbeamertemplate{caption}[numbered]\newcounter{mylastframe}c
%\newcolumntype{Y}{\columncolor[cmyk]{0, 0, 1, 0}\raggedright}
%\newcolumntype{C}{\columncolor[cmyk]{1, 0, 0, 0}\raggedright}
%\newcolumntype{G}{\columncolor[rgb]{0, 1, 0}\raggedright}
%\newcolumntype{R}{\columncolor[rgb]{1, 0, 0}\raggedright}

%\begin{beamerboxesrounded}[upper=uppercol,lower=lowercol,shadow=true]{Block}
%$A = B$.
%\end{beamerboxesrounded}}
\renewcommand{\familydefault}{cmss}
%\usepackage[all]{xy}

\usepackage{tikz}
\usepackage{lipsum}

 \newenvironment{changemargin}[3]{%
 \begin{list}{}{%
 \setlength{\topsep}{0pt}%
 \setlength{\leftmargin}{#1}%
 \setlength{\rightmargin}{#2}%
 \setlength{\topmargin}{#3}%
 \setlength{\listparindent}{\parindent}%
 \setlength{\itemindent}{\parindent}%
 \setlength{\parsep}{\parskip}%
 }%
\item[]}{\end{list}}
\usetikzlibrary{arrows}
%\usepackage{palatino}
%\usepackage{eulervm}
\usecolortheme{lily}

\newtheorem{com}{Comment}
\newtheorem{lem} {Lemma}
\newtheorem{prop}{Proposition}
\newtheorem{thm}{Theorem}
\newtheorem{defn}{Definition}
\newtheorem{cor}{Corollary}
\newtheorem{obs}{Observation}
 \numberwithin{equation}{section}


\title[Causal Inference] % (optional, nur bei langen Titeln nötig)
{Causal Inference}

\author{Justin Grimmer}
\institute[University of Chicago]{Associate Professor\\Department of Political Science \\  University of Chicago}
\vspace{0.3in}

\date{March 28th, 2018}

\begin{document}
\begin{frame}
\titlepage
\end{frame}


\begin{frame}
  \frametitle{Purpose, Scope, and Examples}
Goal in causal inference is to assess the causal effect of some potential cause (e.g. an institution, intervention, policy, or event) on some outcome.\\\bigskip

Examples of such research questions include... \\\medskip

What is the effect of:
\begin{itemize}
 \item political institutions on corruption?
 \item voting technology on  voting fraud?
 \item incumbency status on vote shares?
 \item peacekeeping missions on peace?
 \item mass media on voter preferences?
 \item church attendance on turnout?
\end{itemize}
\end{frame}

\begin{frame}
  \frametitle{What Do We Mean by Causal Inference?}

As in all statistics, we must begin with a model of the reality we are interested in studying, such as:

\[y_i = \alpha + \tau D_i + X_i\beta + \epsilon_i\] \bigskip

Key problems with regression:\medskip \pause
\pause
\begin{itemize}
\item Endogeneity and omitted variable bias\medskip
\item Misspecified functional form \medskip
\item Heterogenous treatment effects \medskip
\end{itemize}

\end{frame}

\begin{frame}{Neyman-Rubin Potential Outcomes Model}

\begin{figure}
\centering
\begin{minipage}{.5\textwidth}
  \centering
  \includegraphics[width=.5\linewidth]{images/Neyman_3.jpeg}
  \caption{Neyman}

\end{minipage}%
\begin{minipage}{.5\textwidth}
  \centering
  \includegraphics[width=.5\linewidth]{images/DonRubin.jpg}
  \caption{Rubin}
\end{minipage}
\end{figure}

\end{frame}

\begin{frame}{Neyman Urn Model}

\centering
  \includegraphics[width=.9\linewidth]{images/potential_outcomes_1.pdf}

\end{frame}

\begin{frame}{Neyman Urn Model}

\centering
  \includegraphics[width=.9\linewidth]{images/potential_outcomes_2.pdf}

\end{frame}

\begin{frame}{Causality with Potential Outcomes}

\begin{definition}[Treatment]
$D_i$: Indicator of treatment intake for {\em unit} $i$
 \[
 D_i = \left\{
 \begin{array}{ll}
  1 & \mbox{if unit $i$ received the treatment}\\
  0 & \mbox{otherwise}.
 \end{array}
 \right.
 \]
\end{definition}

\begin{definition}[Outcome]
 $Y_i$: Observed outcome variable of interest for unit $i$. The treatment occurs temporally before the outcome.
\end{definition}

\begin{definition}[Potential Outcomes]
$Y_{0i}$ and $Y_{1i}$: Potential outcomes for unit $i$
 \[
 Y_{di} = \left\{
 \begin{array}{ll}
  Y_{1i} & \mbox{Potential outcome for unit $i$ with treatment}\\
  Y_{0i} & \mbox{Potential outcome for unit $i$ without treatment}
 \end{array}
 \right.
 \]
\end{definition}

\end{frame}

\begin{frame}{Causality with Potential Outcomes}

\begin{definition}[Causal Effect]
Causal effect of the treatment on the outcome
for unit $i$ is the difference between its two potential outcomes:
\[
\tau_i = Y_{1i} - Y_{0i}
\]
\end{definition}

\begin{assumption}
Observed outcomes are realized as
\[
Y_i = D_i\cdot Y_{1i} + (1-D_i)\cdot Y_{0i}\,\, \mbox{so}\,\,
Y_i = \left\{
 \begin{array}{ll}
  Y_{1i} & \mbox{if $D_i=1$}\\
  Y_{0i} & \mbox{if $D_i=0$}
 \end{array}
 \right.
\]
\end{assumption}

%\pause

%\begin{definition}[Fundamental Problem of Causal Inference]
%Cannot observe both potential outcomes $(Y_{1i},Y_{0i})$
%\end{definition}

\end{frame}

\begin{frame}{Causal Inference as a Missing Data Problem}

\begin{figure}[ht] \centering
    \includegraphics[width = .8 \linewidth]{images/potential_outcome_flowchart.pdf}
\end{figure}

%\emph{Fundamental Problem of Causal Inference}: 

\begin{definition}[Fundamental Problem of Causal Inference]
We cannot observe both potential outcomes. So how can we calculate $\tau_i = Y_{1i} - Y_{0i}$?
\end{definition}

\end{frame}



\begin{frame}{Fundamental Problem of Causal Inference}
\scriptsize

Imagine a study population with 4 units:

\begin{table}[ht]
    \centering
    \begin{tabular}{c   c  c c c }
      $i$ & $D_i$ & $Y_{1i}$ & $Y_{0i}$ & $\tau_i$ \\ \hline
      1  & 1 & 10  & 4 & 6 \\
      2  & 1 & 1  & 2 & -1 \\
      3 & 0 & 3 & 3 & 0 \\
      4 & 0 & 5 & 2 & 3 \\
    \end{tabular}
    \end{table}
    
 What do we observe?   \pause
    
\begin{table}[ht]
    \centering
    \begin{tabular}{c  c c c c c}
      $i$ & $D_i$ & $Y_{1i}$ & $Y_{0i}$ & $\tau_i$ & $Y_i$ \\ \hline
      1  & 1 & 10  & \alert{?}  & \alert{?} & 10 \\
      2  & 1 & 1  & \alert{?}  & \alert{?}  & 1 \\
      3 & 0 & \alert{?}  & 3 & \alert{?}  & 3  \\
      4 & 0 & \alert{?}  & 2 & \alert{?}  & 2 \\
    \end{tabular}
    \end{table}

Causal inference is difficult because it involves
missing data.

\end{frame}






\begin{frame}{Causal Inference as a Missing Data Problem}

How can we calculate $\tau_i = Y_{1i} - Y_{0i}$?\medskip

\begin{itemize}
\itemsep1pt\parskip0pt\parsep0pt
\item
  Homogeneity is one solution:\medskip

  \begin{itemize}
  \itemsep1pt\parskip0pt\parsep0pt
  \item
    If $\{Y_{1i}, Y_{0i}\}$ is constant across individuals, then
    cross-sectional comparisons will recover $\tau_i$\medskip
  \item
    If $\{Y_{1i}, Y_{0i}\}$ is constant across time, then before and
    after comparisons will recover $\tau_i$\medskip
  \end{itemize}
\end{itemize}

\pause

In social phenomena, unfortunately, homogeneity is very rare.

\end{frame}

\begin{frame}{Other Assumptions}
\small
\begin{assumption}
Observed outcomes are realized as
\[
Y_i = D_i\cdot Y_{1i} + (1-D_i)\cdot Y_{0i}
\]
\end{assumption}
\pause
\begin{itemize}
\itemsep1pt\parskip0pt\parsep0pt
\item
  Embedded in this formulation is the assumption that potential outcomes
  for unit $i$ are unaffected by treatment assignment for unit $j$.
\end{itemize}

\begin{itemize}
\itemsep1pt\parskip0pt\parsep0pt
\item
  Assumption known by several names:

  \begin{itemize}
  \itemsep1pt\parskip0pt\parsep0pt
  \item
    \textbf{S}table \textbf{U}nit \textbf{T}reatment \textbf{V}alue
    \textbf{A}ssumption (SUTVA)
  \item
    No interference
  \item
    Individualized Treatment Response
  \end{itemize}\medskip
\item
  Examples: vaccination, fertilizer on plot yield, communication
\end{itemize}

\end{frame}

\begin{frame}{Potential Outcomes with Interference}
\small
Let $\mathbf{D}=\{D_i,D_j\}$ be the set of vectors of treatment assignments for
two units $i$ (me) and $j$ (you).\\\medskip

How many elements in $\mathbf{D}$?

\pause

\[
\mathbf{D}=\{(D_i=0,D_j=0),(D_i=1,D_j=0),(D_i=0,D_j=1),(D_i=1,D_j=1)\}
\]

\pause

How many potential outcomes for unit $i$? 

\pause

\begin{small}
\[
Y_{1i}(\mathbf{D}) = \left\{
 \begin{array}{l}
  \textcolor{blue}{Y_{1i}(1,1)} \\
  \textcolor{cyan}{Y_{1i}(1,0)}
 \end{array}
 \right.\,\,
Y_{0i}(\mathbf{D}) = \left\{
 \begin{array}{l}
  \textcolor{red}{Y_{0i}(0,1)} \\
  \textcolor{orange}{Y_{0i}(0,0)}
 \end{array}
 \right.
\]
\end{small}

\end{frame}

\begin{frame}{Potential Outcomes with Interference}
\small
How many causal effects for unit $i$?

\pause

\begin{small}
\[
\tau_i(\mathbf{D}) = \left\{
 \begin{array}{l}
  \textcolor{blue}{Y_{1i}(1,1)} - \textcolor{orange}{Y_{0i}(0,0)} \\
 \textcolor{blue}{ Y_{1i}(1,1)} - \textcolor{red}{Y_{0i}(0,1)} \\
  \textcolor{cyan}{Y_{1i}(1,0)} - \textcolor{orange}{Y_{0i}(0,0)} \\
\textcolor{cyan}{  Y_{1i}(1,0)} - \textcolor{red}{Y_{0i}(0,1)} \\
 \textcolor{blue}{ Y_{1i}(1,1)} - \textcolor{cyan}{Y_{1i}(1,0)} \\
 \textcolor{red}{ Y_{0i}(0,1)}  - \textcolor{orange}{Y_{0i}(0,0)} \\
 \end{array}
 \right.
\]
\end{small}\medskip

\pause

How many potential outcomes are observed for unit $i$?\\\medskip \pause Since we only observe one of the four potential outcomes, the missing data problem for causal inference is even more severe.

\end{frame}

\begin{frame}{Potential Outcomes with Interference}
\small
The No Interference assumption states that unit $i$'s potential outcomes
depend on $D_i$, not $\mathbf{D}$:\medskip

$\textcolor{blue}{Y_{1i}(1,1)}=\textcolor{cyan}{Y_{1i}(1,0)}$ and
$\textcolor{red}{Y_{0i}(0,1)}=\textcolor{orange}{Y_{0i}(0,0)}$\medskip

This assumption furthermore allows us to define the effect for unit $i$
as $\tau_i = Y_{1i} - Y_{0i}$.\medskip

%\pause

%More generally, if $N_T$ units receive treatment, then there are
%${N \choose N_T}$ possible treatment allocations. Thus, each unit will
%have ${N \choose N_T}$ potential outcomes.\medskip

\pause

No interference is an example of an \textbf{exclusion restriction}. We
rely on outside information to rule out the possibility of certain
causal effects (e.g. you taking the treatment has no effect on my
potential outcomes).\medskip

Note that traditional models like regression also involve an implicit SUTVA assumption ($Y_i$ depends on $X_i$)

\end{frame}

\begin{frame}{Potential Outcomes with Interference}
\small
Some Examples of Interference:

\begin{itemize}
\itemsep1pt\parskip0pt\parsep0pt
\item  Contagion
\item  Displacement
\item  Communication
%\item   Social comparison
\item   Deterrence
%\item  Persistence and memory
\end{itemize}

Causal inference in the presence of interference between subjects is an
area of active research. Specially tailored experimental designs have
been developed to study these interactions, e.g.~Miguel and Kremer
(2004) and Sinclair, McConnell, and Green (2012).

\end{frame}

\begin{frame}{Back to the Neyman Urn Model}

\centering
  \includegraphics[width=.9\linewidth]{images/potential_outcomes_1.pdf}

\end{frame}

\begin{frame}{Estimands}
\small
Because $\tau_i$ are unobservable, we shift what we are interested in to:

\begin{definition}[Average Treatment Effect (ATE)]

   
    {\centering
 $\tau_{ATE} =$  Average of all treatment potential outcomes $-$\\ Average of all control potential outcomes
    
    or  
    $$\tau_{ATE} = \frac{\sum_i^N Y_{1i}}{N} - \frac{\sum_i^N Y_{0i}}{N} $$

    or
    $$ \tau_{ATE} = E[Y_{1i} - Y_{0i}]$$
    or
    $$ \tau_{ATE} = E[\tau_i]$$
    }
\end{definition}

\end{frame}

\begin{frame}{Other Estimands}
\small
\begin{definition}[Average treatment effect on the treated (ATT)]

$$ \tau_{ATT} = E[Y_{1i} - Y_{0i} | D_i = 1]$$

\end{definition}

\pause

\begin{definition}[Average treatment effect on the controls (ATC)]

$$ \tau_{ATC} = E[Y_{1i} - Y_{0i} | D_i = 0]$$

\end{definition}

\pause

\begin{definition}[Average treatment effects for subgroups]
    {\centering
    $$ \tau_{ATE(X)} = E[Y_{1i} - Y_{0i} | X_i = x]$$
    or 
        $$ \tau_{ATT(X)} = E[Y_{1i} - Y_{0i} | D_i = 1, X_i = x]$$
}
\end{definition}

\end{frame}

\begin{frame}{Average Treatment Effect}

Imagine a study population with 4 units:

\begin{table}[ht]
    \centering
    \begin{tabular}{c  c c c c }
      $i$ & $D_i$ & $Y_{1i}$ & $Y_{0i}$ & $\tau_i$ \\ \hline
      1  & 1 & 10  & 4 & 6 \\
      2  & 1 & 1  & 2 & -1 \\
      3 & 0 & 3 & 3 & 0 \\
      4 & 0 & 5 & 2 & 3 \\
    \end{tabular}
    \end{table}

What is the ATE?\medskip

\pause

$E[Y_{1i} - Y_{0i}] = 1/4 \times (6 + -1 + 0 + 3) = 2$\medskip

\pause

Note: Average effect is positive, but $\tau_i$ are negative for some
units!

\end{frame}

\begin{frame}{Average Treatment Effect on the Treated}

Imagine a study population with 4 units:

\begin{table}[ht]
    \centering
    \begin{tabular}{c  c c c c }
      $i$ & $D_i$ & $Y_{1i}$ & $Y_{0i}$ & $\tau_i$ \\ \hline
      1  & 1 & 10  & 4 & 6 \\
      2  & 1 & 1  & 2 & -1 \\
      3 & 0 & 3 & 3 & 0 \\
      4 & 0 & 5 & 2 & 3 \\
    \end{tabular}
    \end{table}
    
    \medskip

What is the ATT and ATC?\medskip

\pause

$E[Y_{1i} - Y_{0i} | D_i = 1] = 1/2 \times (6 + -1) = 2.5$\\\medskip
$E[Y_{1i} - Y_{0i} | D_i  = 0] = 1/2 \times (0 + 3) = 1.5$

\end{frame}

%\begin{frame}{Naive Comparison: Difference in Means}
%\small
%Comparisons between \emph{observed} outcomes of treated and control
%units can often be misleading.
%
%\begin{equation}
%\begin{split}
%E[Y|D=1]-E[Y_i|D_i=0]=E[Y_{1i} | D_i=1]-E[Y_{0i} | D_i=0]\\
%                 =\underbrace{E[Y_{1i} - Y_{0i} | D_i=1]}_{\mbox{ATT}}
%                 +\underbrace{\{ E[Y_{i0} | D_i=1]- E[Y_{0i} | D_i=0]\}}_{\mbox{BIAS}}\nonumber
%\end{split}
%\end{equation}
%
%\begin{itemize}
%\itemsep1pt\parskip0pt\parsep0pt
%\item
%  Bias term unlikely to be 0 in most applications.
%\item
%  Selection into treatment is often associated with the potential
%  outcomes
%\end{itemize}
%
%\end{frame}

\begin{frame}{Naive Comparison: Difference in Means}
\small
Comparisons between \emph{observed} outcomes of treated and control
units can often be misleading.

\begin{scriptsize}
\begin{align}
\begin{split}
E[Y_i|D=1]-E[Y_i|D_i=0] \\
&=E[Y_{1i} | D_i=1]-E[Y_{0i} | D_i=0]\\
                 &=\underbrace{E[Y_{1i} - Y_{0i} | D_i=1]}_{\mbox{ATT}}
                 +\underbrace{\{ E[Y_{i0} | D_i=1]- E[Y_{0i} | D_i=0]\}}_{\mbox{BIAS}}\nonumber
\end{split}
\end{align}
\end{scriptsize}

\begin{itemize}
\itemsep1pt\parskip0pt\parsep0pt
\item
  Bias term unlikely to be 0 in most applications.
\item
  Selection into treatment is often associated with the potential
  outcomes.
\end{itemize}

\end{frame}

\begin{frame}{Selection Bias}
\small
\begin{scriptsize}
\begin{align}
\begin{split}
E[Y_i|D_i=1]-E[Y_i|D_i=0] \\
&=E[Y_{1i} | D_i=1]-E[Y_{0i} | D_i=0]\\
                 &=\underbrace{E[Y_{1i} - Y_{0i} | D_i=1]}_{\mbox{ATT}}
                 +\underbrace{\{ E[Y_{i0} | D_i=1]- E[Y_{0i} | D_i=0]\}}_{\mbox{BIAS}}\nonumber
\end{split}
\end{align}
\end{scriptsize}

Example: Church Attendance and Political Participation

\begin{itemize}
\itemsep1pt\parskip0pt\parsep0pt
\item
  Churchgoers are likely to differ from non-churchgoers on a range of
  background characteristics (e.g.~civic duty).\medskip
\item
  Given these differences, turnout for churchgoers would be higher than
  for non-churchgoers even if churchgoers never attended church or
  church had zero mobilizing effect
  ($E[Y_0 | D=1]-E[Y_0 | D=0]>0$).
\end{itemize}

\end{frame}

\begin{frame}{Selection Bias}

\small
\begin{scriptsize}
\begin{align}
\begin{split}
E[Y_i|D_i=1]-E[Y_i|D_i=0] \\
&=E[Y_{1i} | D_i=1]-E[Y_{0i} | D_i=0]\\
                 &=\underbrace{E[Y_{1i} - Y_{0i} | D_i=1]}_{\mbox{ATT}}
                 +\underbrace{\{ E[Y_{i0} | D_i=1]- E[Y_{0i} | D_i=0]\}}_{\mbox{BIAS}}\nonumber
\end{split}
\end{align}
\end{scriptsize}

Example: Gender Quotas and Redistribution Towards Women

\begin{itemize}
\itemsep1pt\parskip0pt\parsep0pt
\item
  Countries with gender quotas are likely countries where women are
  politically mobilized.\medskip
\item
  Given this difference, policies targeted towards women are more common in
  quota countries even if these countries had not adopted quotas
  ($E[Y_0 | D=1]-E[Y_0 | D=0]>0$).
\end{itemize}

\end{frame}


\begin{frame}
  \frametitle{Regression to Estimate the Average Treatment Effect}
\small

What happens when you run a regression of the observed outcome on the treatment indicator to estimate the ATE?\\\bigskip

The ATE can be expressed as a regression equation:

\begin{eqnarray*}
% \nonumber to remove numbering (before each equation)
  Y_i &=& D_i\, Y_{1i} + (1-D_i)\, Y_{0i} \\
   &=& Y_{0i} + (Y_{1i}- Y_{0i})\,D_i \\
   &=& \underbrace{\bar Y_0}_{\alpha} + \underbrace{(\bar Y_1 - \bar Y_0)}_{\tau_{Reg}} D_i + \underbrace{\{(Y_{i0} - \bar Y_0) + D_i \cdot [(Y_{i1} - \bar Y_1) - (Y_{i0} - \bar Y_0)] \}}_{\epsilon} \\ 
%   &=&\E[Y_0] + (E[Y_1]-\E[Y_0])\,D_i + u_i \\
%   &=& \beta + \alpha_{ATE} D_i + u_i
   &=& \alpha + \tau_{Reg} D_i + \epsilon_i
 \end{eqnarray*}
 
%where the disturbance term is $u_i\equiv Y_{0i}-\E[Y_0] +[(Y_{1i}-\E[Y_1])-(Y_{0i}-\E[Y_0])]D_i$\\
\begin{itemize}
%\item  $D_i$ is random in this interpretation, as opposed to classical regression.
\item   $\tau_{Reg}$ could be biased and inconsistent for $\tau_{ATE}$ in two ways: \pause
  \begin{itemize}
  \item Baseline difference in potential outcomes under control that is correlated with $D_i$. 
  \item Individual treatment effects $\tau_i$ are correlated with $D_i$ 
  \end{itemize}\pause
\item
  Effect heterogeneity implies ``heteroskedasticity'', i.e. error variance differs by values of $D_i$.
  \begin{itemize}
  \item  Neyman model imples ``robust'' standard errors. 
  \end{itemize}
  
\end{itemize}

\end{frame}


\begin{frame}{The Assignment Mechanism}
\small
\begin{itemize}
\itemsep1pt\parskip0pt\parsep0pt
\item
  Since missing potential outcomes are unobservable we must make
  assumptions to fill them in, i.e. \textbf{estimate} missing potential
  outcomes.
\item
  In the causal inference literature, we typically make assumptions
  about the \textbf{assignment mechanism} to do so.
\end{itemize}

\pause 

\begin{definition}[Assignment Mechanism]
Assignment mechanism is the procedure that determines which units
are selected for treatment. Examples include:
 \begin{itemize}
\item random assignment
\item selection on observables
\item selection on unobservables
\end{itemize}
\end{definition}

\begin{itemize}
\itemsep1pt\parskip0pt\parsep0pt
\item
  Most statistical models of causal inference attain identification of
  treatment effects by restricting the assignment mechanism in some way.
\end{itemize}
\end{frame}


\begin{frame}{Assignment Mechanism}

Imagine a study population with 4 units:

\begin{table}[ht]
    \centering
    \begin{tabular}{c c c c c c }
      $i$ &$\Pr(D_i = 1)$ & $D_i$ & $Y_{1i}$ & $Y_{0i}$ & $\tau_i$ \\ \hline
      1  &\color{red}{?} &1 & 10  & 4 & 6 \\
      2  &\color{red}{?} &1 & 1  & 2 & -1 \\
      3 &\color{red}{?} &0 & 3 & 3 & 0 \\
      4 &\color{red}{?} &0 & 5 & 2 & 3 \\
    \end{tabular}
    \end{table}

\end{frame}


\begin{frame}

\centering
\fbox{No causation without manipulation?}\bigskip\bigskip\bigskip

\pause

Always ask:\\\bigskip
What is the \alert{ideal experiment} you would run if you had infinite
resources and power?

\end{frame}

\begin{frame}{Causal Inference Workflow}

\centering
  \includegraphics[width=.9\linewidth]{images/ciworkflow.pdf}

\end{frame}


\begin{frame}{Summing Up: Neyman-Rubin causal model}
\small

\begin{itemize}
\itemsep1pt\parskip0pt\parsep0pt
\item
  Useful for studying the ``effects of causes,'' less so for the
  ``causes of effects.''\medskip
\item
  No assumption of homogeneity, allows for causal effects to vary unit
  by unit.\medskip

  \begin{itemize}
  \itemsep1pt\parskip0pt\parsep0pt
  \item
    No single ``causal effect,'' thus the need to be precise about the
    target estimand.
  \end{itemize}\medskip
\item
  Distinguishes between \emph{observed} outcomes and \emph{potential}
  outcomes.\medskip
\item
  Causal inference is a missing data problem: we typically make
  assumptions about the assignment mechanism to go from descriptive
  inference to causal inference.
\end{itemize}

\end{frame}

\begin{frame}{Alternative Causal Models}
\small
The Neyman-Rubin causal model is popular in the social and health sciences, but
alternatives exist:

\includegraphics[width=.7\linewidth]{images/dag.png}

\begin{itemize}
\itemsep1pt\parskip0pt\parsep0pt
\item
  Structural Equation Modeling:

  \begin{itemize}
  \itemsep1pt\parskip0pt\parsep0pt
  \item
    Write down causal model using Directed Acyclic Graphs (DAG)
  \item
    Causal effects are defined by interventions that set variables to
    specified values in the causal model.
  \item
    Set of axioms (``Do Calculus'') that establish identifiablity of
    causal parameters given structure of the causal graph.\\
  \item
    Can be re-expressed in potential outcome notation (though sometimes
    difficult!)
  \end{itemize}
\item
  Causality without Counterfactuals (Dawid 2000)
\end{itemize}

\end{frame}


\end{document}