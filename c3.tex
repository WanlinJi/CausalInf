\documentclass{beamer}

%\usepackage[table]{xcolor}
\mode<presentation> {
  \usetheme{Boadilla}
%  \usetheme{Pittsburgh}
%\usefonttheme[2]{sans}
\renewcommand{\familydefault}{cmss}
%\usepackage{lmodern}
%\usepackage[T1]{fontenc}
%\usepackage{palatino}
%\usepackage{cmbright}
  \setbeamercovered{transparent}
\useinnertheme{rectangles}
}
%\usepackage{normalem}{ulem}
%\usepackage{colortbl, textcomp}
\setbeamercolor{normal text}{fg=black}
\setbeamercolor{structure}{fg= black}
\definecolor{trial}{cmyk}{1,0,0, 0}
\definecolor{trial2}{cmyk}{0.00,0,1, 0}
\definecolor{darkgreen}{rgb}{0,.4, 0.1}
\usepackage{array}
\beamertemplatesolidbackgroundcolor{white}  \setbeamercolor{alerted
text}{fg=red}
\newtheorem{assumption}{Assumption}

\setbeamertemplate{caption}[numbered]\newcounter{mylastframe}

%\usepackage{color}
\usepackage{tikz}
\usetikzlibrary{arrows}
\usepackage{colortbl}
%\usepackage[usenames, dvipsnames]{color}
%\setbeamertemplate{caption}[numbered]\newcounter{mylastframe}c
%\newcolumntype{Y}{\columncolor[cmyk]{0, 0, 1, 0}\raggedright}
%\newcolumntype{C}{\columncolor[cmyk]{1, 0, 0, 0}\raggedright}
%\newcolumntype{G}{\columncolor[rgb]{0, 1, 0}\raggedright}
%\newcolumntype{R}{\columncolor[rgb]{1, 0, 0}\raggedright}

%\begin{beamerboxesrounded}[upper=uppercol,lower=lowercol,shadow=true]{Block}
%$A = B$.
%\end{beamerboxesrounded}}
\renewcommand{\familydefault}{cmss}
%\usepackage[all]{xy}

\usepackage{tikz}
\usepackage{lipsum}

 \newenvironment{changemargin}[3]{%
 \begin{list}{}{%
 \setlength{\topsep}{0pt}%
 \setlength{\leftmargin}{#1}%
 \setlength{\rightmargin}{#2}%
 \setlength{\topmargin}{#3}%
 \setlength{\listparindent}{\parindent}%
 \setlength{\itemindent}{\parindent}%
 \setlength{\parsep}{\parskip}%
 }%
\item[]}{\end{list}}
\usetikzlibrary{arrows}
%\usepackage{palatino}
%\usepackage{eulervm}
\usecolortheme{lily}

\newtheorem{com}{Comment}
\newtheorem{lem} {Lemma}
\newtheorem{prop}{Proposition}
\newtheorem{thm}{Theorem}
\newtheorem{defn}{Definition}
\newtheorem{cor}{Corollary}
\newtheorem{obs}{Observation}
 \numberwithin{equation}{section}


\title[Causal Inference] % (optional, nur bei langen Titeln nötig)
{Causal Inference}

\author{Justin Grimmer}
\institute[University of Chicago]{Associate Professor\\Department of Political Science \\  University of Chicago}
\vspace{0.3in}

\date{March 30th, 2018}

\begin{document}
\begin{frame}
\titlepage
\end{frame}



\section{Random Sampling}

\begin{frame}{Population}
\scriptsize
\begin{definition}[Population]
Consider a population with $N$ units that have fixed values of an attribute $X_1,X_2,...,X_N$.\\\medskip $X_i$ is the fixed value of the attribute for the i-th member of the population.\\\bigskip

The population mean is given by $$\mu=\frac{1}{N} \sum_i^N X_i$$ and the population variance is given by $$\sigma^2 = \frac{1}{N}   \sum_i^N (X_i-\mu)^2$$ 
\end{definition}

\end{frame}

\begin{frame}{Simple Random Sampling}
\scriptsize

\begin{definition}[Random Sample with replacement]
Assume a sample of $n$ elements that are drawn with replacement and the attribute values of the sample members are denoted by $x_1,x_2,x_3,...,x_n$. Each $x_i$ is a random variable, i.e. the value of the i-th member of the sample. Sampling with replacement is equivalent to iid sampling. The draws $x_1,...,x_n$ are independent and identically distributed random variables. 

\end{definition}

\begin{definition}[Random Sample without replacement]
Assume a sample of $n$ elements that are drawn without replacement and the attribute values of the sample members are denoted by $x_1,x_2,x_3,...,x_n$. Each $x_i$ is a random variable, i.e. the value of the i-th member of the sample. 


\end{definition}

\begin{itemize}
\item For sampling without replacement, each of the ${N \choose n}$ possible samples of size $n$ taken without replacement has the same probability of occurrence
\item Sampling without replacement is fairly similar to the iid sampling with replacement, except that the draws $x_1,...,x_n$ are no longer independent.
\begin{itemize}
\item \scriptsize The covariance between two draws $x_i$ and $x_j$ is  $Cov[x_i,x_j]=-\frac{\sigma^2}{(N-1)} \mbox{ if }  i\neq j$
\end{itemize}
\item Formulas for sampling with replacement still offer a good approximation as long as the sample size $n$ is small relative to the population size $N$
\end{itemize}

\end{frame}

\begin{frame}{Expected Value and Variance of each Sample Member}
\scriptsize

\begin{lemma}[Expected Value and Variance of each Sample Member]
Write distinct attribute values of the population members by $\theta_1,\theta_2,...,\theta_m$ and denote the number of population members that have the value $\theta_j$ by $n_j$. Then $x_i$ is a discrete random variable with a simple PMF which is
$$
P(x_i=\theta_j)=\frac{n_j}{N}\,\,\, \mbox{with} \,\,\, E[x_i]=\mu \,\,\, \mbox{and} \,\,\, Var[x_i]=\sigma^2
$$\\
This holds for sampling with and without replacement.
\end{lemma}

\end{frame}

\begin{frame}{Expected Value and Variance of each Sample Member}
\scriptsize

\begin{proof}\tiny
$x_i$ can only take on possible values $\theta_1,\theta_2,...,\theta_m$ and since each population member is equally likely to be the i-th member in the sample it follows that the probability that $x_i$ assumes the value $\theta_j$ is $\frac{n_j}{N}$. The expected value then is

\begin{eqnarray}
E[x_i]&=&\sum_{j=1}^{m} \theta_j P(x_i=\theta_j)\\
  &=&   \sum_{j=1}^{m} \theta_j  \frac{n_j}{N} \\
  &=&  \frac{1}{N} \sum_{j=1}^{m}n_j\theta_j\\
  &=&  \frac{1}{N} N \mu\\
  &=& \mu \\
\end{eqnarray}

the second to last equality follows because there are $n_j$ population members with value $\theta_j$ and therefore the sum is equal to the sum of the values of all population members.\\
\end{proof}
\end{frame}


\begin{frame}{Expected Value and Variance of each Sample Member}
\scriptsize

\begin{proof}\tiny
The variance of each sample member is

\begin{eqnarray}
Var[x_i] &=& E[x_i^2]-[E[x_i]]^2 \\
  &=&  \sum_{j=1}^{m}\theta_j^2P(x_i=\theta_j) - \mu^2\\
  &=& \frac{1}{N} \sum_{j=1}^{m}n_j\theta_j^2 - \mu^2\\
    &=& \frac{1}{N}  \sum_i^N X_i^2 - \mu^2 \\
    &=&  \frac{1}{N}    ( \sum_i^N X_i^2 - 2N\mu^2 + N \mu^2 )  \\
    &=&  \frac{1}{N}    ( \sum_i^N X_i^2 - 2\mu \sum_i^N X_i + \sum_i^N \mu^2 )\\
   &=& \frac{1}{N}   \sum_i^N (X_i-\mu)^2\\ 
    &=& \sigma^2\\
\end{eqnarray}
\end{proof}
\end{frame}


\begin{frame}{Sample Mean Estimator}
\scriptsize

\begin{theorem}[Sample Mean Unbiased for Population Mean]
Let $\bar{x}$ be the sample mean estimator given by $\bar{x}= \frac{1}{n} \sum_{i=1}^n x_i$. Then for simple random sampling we have that the sample mean is an unbiased estimator of the population mean
$$
E[\bar{x}]=\mu
$$
\end{theorem}
\begin{proof}
Using the lemma $E[x_i]=\mu$ we have 
$$
E[\bar{x}] = \frac{1}{n} E[\sum_{i=1}^n x_i]=\frac{1}{n} n\, \mu=\mu
$$
This holds for sampling with and without replacement.
\end{proof}
\end{frame}

\begin{frame}{Variance of the Sample Mean}
\scriptsize

\begin{theorem}[Variance of the Sample Mean]
For simple random sampling with replacement the variance of the sample mean is

$$
Var[\bar{x}]=\frac{\sigma^2}{n}
$$
For simple random sampling without replacement the variance of the sample mean is
\begin{eqnarray}
Var[\bar{x}]&=&\frac{\sigma^2}{n} \left( \frac{N-n}{N-1} \right)\\
  &=& \frac{\sigma^2}{n} \left(1- \frac{n-1}{N-1} \right) 
\end{eqnarray}
where $\left(1- \frac{n-1}{N-1} \right)$ is often called the finite population correction.\\\bigskip

If the sampling fraction $\frac{n}{N}$ is small, the finite population correction is close to one and therefore the variance with replacement is a good approximation. If $n$ grows close to $N$ the variance goes to zero.
\end{theorem}
\end{frame}

\begin{frame}{Variance of the Sample Mean}
\scriptsize

\begin{proof}
Recall that for a linear combination of random variables the following result holds:
$$
Var[a+\sum_{i=1}^n b_i x_i] = \sum_{i=1}^n \sum_{j=1}^n b_i b_j Cov[x_i,x_j]
$$
where a and b are constants. Since the sample mean is a linear combination $\bar{x}= \frac{1}{n} \sum_{i=1}^n x_i$ we have that 

$$
Var[\bar{x}_i]=\frac{1}{n^2} \sum_{i=1}^n \sum_{j=1}^n  Cov[x_i,x_j]
$$
If we sample with replacement the $x_i$ draws are independent and therefore $ Cov[x_i,x_j]=0$, but  $Cov[x_i,x_i]=Var[x_i]=\sigma^2$ and therefore we would get  the usual variance
\begin{eqnarray}
Var[\bar{x}_i]&=&\frac{1}{n^2} \sum_{i=1}^n Var[x_i]  \\
  &=& \frac{\sigma^2}{n}
\end{eqnarray}
\end{proof}

\end{frame}

\begin{frame}{Variance of the Sample Mean}
\scriptsize

\begin{proof}

Sampling without replacement induces a dependence among the $x_i$ such that the covariance between two draws is given by
$$
Cov[x_i,x_j]=-\frac{\sigma^2}{(N-1)} \mbox{ if }  i\neq j
$$
Using this result we can derive the variance for the sample mean for sampling without replacement as
\begin{eqnarray}
Var[\bar{x}_i]&=&\frac{1}{n^2} \sum_{i=1}^n \sum_{j=1}^n  Cov[x_i,x_j] \\
  &=& \frac{1}{n^2}     \sum_{i=1}^n Var[x_i] +  \frac{1}{n^2}     \sum_{i=1}^n \sum_{i \neq j}  Cov[x_i,x_j] \\
  &=& \frac{\sigma^2}{n} -  \frac{1}{n^2}  n(n-1) \frac{\sigma^2}{N-1}  \\
  &=&  \frac{\sigma^2}{n} \left(1- \frac{n-1}{N-1} \right) 
\end{eqnarray}
\end{proof}
\end{frame}



\begin{frame}{Unbiased Estimator for Population Variance}
\scriptsize
\begin{theorem}[Unbiased Estimator for Population Variance]
Recall that the population variance is given by 
$$
\sigma^2 = \frac{1}{N}   \sum_i^N (X_i-\mu)^2
$$

Typically we use a sample analog estimator and estimate the population variance by the simple sample variance $$\hat{\sigma}_n^2 = \frac{1}{n}   \sum_i^n (x_i-\bar{x})^2 $$ For sampling with replacement this estimator is biased since $E[\hat{\sigma}_n^2 ]=\frac{n-1}{n}\sigma^2$.\\\bigskip

So an unbiased estimator can be obtained by multiplying $\hat{\sigma}_n^2$ with a factor of $\frac{n}{n-1}$ (often called the Bessel correction) which leads to the commonly used modified sample variance estimator $$\hat{\sigma}_{n-1}^2 = \frac{1}{n-1}   \sum_i^n (x_i-\bar{x})^2 $$ which is unbiased so that $E[\hat{\sigma}_{n-1}^2 ]=\sigma^2$. 

\end{theorem}

\end{frame}

\begin{frame}{Unbiased Estimator for Population Variance}
\scriptsize
\begin{proof}
\begin{eqnarray}
  E[\sigma_n^2] &=& E\left[ \frac 1n \sum_{i=1}^n \left(x_i - \frac 1n \sum_{j=1}^n x_j \right)^2 \right] \\ 
  & =& \frac 1n \sum_{i=1}^n E\left[ x_i^2 - \frac 2n x_i \sum_{j=1}^n x_j + \frac{1}{n^2} \sum_{j=1}^n x_j \sum_{k=1}^n x_k \right] \\ 
  & = &\frac 1n \sum_{i=1}^n \left[ \frac{n-2}{n} E[x_i^2] - \frac 2n \sum_{j \neq i} E[x_i x_j] + \frac{1}{n^2} \sum_{j=1}^n \sum_{k \neq j}^n E[x_j x_k] +\frac{1}{n^2} \sum_{j=1}^n E[x_j^2] \right] \\ 
  & =& \frac 1n \sum_{i=1}^n \left[ \frac{n-2}{n} (\sigma^2+\mu^2) - \frac 2n (n-1) \mu^2 + \frac{1}{n^2} n (n-1) \mu^2 + \frac 1n (\sigma^2+\mu^2) \right] \\ 
  & = &\frac{n-1}{n} \sigma^2. \\
\end{eqnarray}
\end{proof}
\end{frame}


\begin{frame}{Unbiased Estimator for the Variance of the Sample Mean}

\scriptsize

\begin{theorem}[Unbiased Estimator for the Variance of the Sample Mean]
Recall that for sampling with replacement the variance of the sample mean is given by
$$
Var[\bar{x}_i]= \frac{\sigma^2}{n}
$$
And the modified sample variance 
$$
\hat{\sigma}_{n-1}^2 = \frac{1}{n-1}   \sum_i^n (x_i-\bar{x})^2 
$$
is an unbiased estimator for the population variance so that $E[\hat{\sigma}_{n-1}^2 ]=\sigma^2$.\\\bigskip

So by plugging in we obtain an unbiased estimator for the variance of the sample mean

$$
Var[ \bar{x}] =  \frac{\hat{\sigma}_{n-1}^2}{n}
$$
\end{theorem}

\end{frame}



\begin{frame}{Unbiased Estimator for Population Variance}
\scriptsize

\begin{theorem}[Unbiased Estimator for Population Variance]

For simple random sampling without replacement, the population variance is the same
$$
\sigma^2 = \frac{1}{N}   \sum_i^N (X_i-\mu)^2
$$
but it turns out that 
$$
E[\hat{\sigma}_n^2]=\sigma^2 \left( \frac{n-1}{n} \right) \frac{ N}{N-1}
$$
so the simple sample variance estimator $\hat{\sigma}_n^2$ is also biased for the population variance. However, an unbiased estimator can be obtained by multiplying $\hat{\sigma}_n^2$ by $\frac{n}{n-1} \frac{N-1}{N}$.% which is equivalent to
%$$
%\hat{\sigma}_n^2 \, \frac{n}{n-1} \frac{N-1}{N} =  \frac{1}{n}  \sum_i^n (x_i-\bar{x})^2 \,\frac{n}{n-1} \frac{N-1}{N} = \frac{1}{n-1}   \sum_i^n (x_i-\bar{x})^2  \frac{N-1}{N} =  \hat{\sigma}_{n-1}^2  \frac{N-1}{N} 
%$$
\end{theorem}
\end{frame}

\begin{frame}{Unbiased Estimator for Population Variance}
\tiny
\begin{proof}
\begin{eqnarray}
\hat{\sigma}_n^2 &=&  \frac{1}{n}   \sum_i^n (x_i-\bar{x})^2  =  \frac{1}{n}   \sum_i^n x_i^2 - \bar{x}^2  \\ 
E[\hat{\sigma}_n^2 ]  & =& \frac{1}{n}  \sum_{i=1}^n  E[ x_i^2] - E[\bar{x}^2] 
\end{eqnarray}
Now we know that 
\begin{eqnarray}
E[x_i^2] &=& Var[x_i] + [E[x_i]]^2  = \sigma^2 + \mu^2  
\end{eqnarray}\vspace{-.1in}
\begin{eqnarray}
E[\bar{x}^2] &=& Var[\bar{x}] + [E[\bar{x}]]^2  = \frac{\sigma^2}{n} \left(1- \frac{n-1}{N-1} \right)   + \mu^2   
\end{eqnarray}
so substituting in these expressions we get
\begin{eqnarray}
E[\hat{\sigma}_n^2 ]  &=& \frac{1}{n}  \sum_{i=1}^n  E[ x_i^2] - E[\bar{x}^2] \\ 
 &=& \frac{1}{n}  \sum_{i=1}^n \sigma^2 + \mu^2  - \left( \frac{\sigma^2}{n} \left(1- \frac{n-1}{N-1} \right)   + \mu^2 \right)   \\ 
 & =& \sigma^2 \left( \frac{n-1}{n} \right) \frac{ N}{N-1}  
\end{eqnarray}
\end{proof}
\end{frame}


\begin{frame}{Unbiased Estimator for the Variance of the Sample Mean}

\scriptsize

\begin{theorem}[Unbiased Estimator for the Variance of the Sample Mean]
Recall that for sampling without replacement the variance of the sample mean is given by
$$
Var[\bar{x}]=\frac{\sigma^2}{n} \left( \frac{N-n}{N-1} \right)
$$

So by plugging in the unbiased estimator of the population variance, we obtain an unbiased estimator of the variance of the sample mean is given by

\begin{eqnarray}
\hat{\sigma}^2_{\bar{x}}&=& \frac{\hat{\sigma}_n}{n}    \left( \frac{N-n}{N-1} \right) \frac{n}{n-1} \frac{N-1}{N}   \\
  &=& \frac{\hat{\sigma}_{n-1}^2}{n} \left( 1- \frac{n}{N}\right) 
\end{eqnarray}
where $\hat{\sigma}_{n-1}^2 = \frac{1}{n-1}   \sum_i^n (x_i-\bar{x})^2$. So the variance estimator is the same as under sampling with replacement but multiplies it with the finite population correction $\left( 1- \frac{n}{N}\right)$.
\end{theorem}
\end{frame}


\begin{frame}{Types of Randomization}

\small

\begin{enumerate}
\item Simple random assignment
\begin{itemize}
\item For each unit $i$ we flip a coin, ie. randomly sample from a Bernoulli distribution where the probability of success is $p$
\item Equivalent to random sampling with replacement: the treatment assignment for unit, $D_i$, is a random variable with $P(D_i=1)=p$ so each unit has the same probability of receiving the treatment
\item The vector of treatment assignments for all the units $D_1,D_2,...,D_N$ constitute an iid sample
\end{itemize}
\item Complete random assignment

\begin{itemize}
\item $N_1$ of $N$ units are randomly assigned to treatment and $N_0$ units to control (with $N= N_1 + N_0$)
\item Equivalent to random sampling without replacement: each unit has the same probability of receiving the treatment, $p=N_1/N$, because each vector of treatment assignments $N \choose N_1$ is equally likely
\item The treatment assignments for the units $D_1,D_2,...,D_N$ are not independent 
\end{itemize}
\end{enumerate}

\end{frame}

\begin{frame}{Randomization for Causal Inference}
\small
\begin{itemize}
\item Random assignment allows for unbiased estimation of missing counterfactuals, such the unobserved average outcome under treatment $E[Y_{1i}]$, just like random sampling allows for unbiased estimation of the population mean.
\item All units have a potential outcome under treatment $Y_{1i}$ and we need to find the unobserved average outcome under treatment $E[Y_{1i}]$
\item For the units that are assigned to the treatment we observe $Y_{1i}=Y_i$ 
\item So if the units are randomly assigned to the treatment, this is like randomly sampling some $Y_{1i}$s from the population of all $Y_{1i}$s
\item So the average observed outcome of the randomly chosen treated units will be an unbiased and consistent estimator of the average  outcome under treatment in the population of all units $E[Y_{1i}]$
\item This will allows us to identify the average treatment effects $\tau_{ATE}=E[Y_{1i}]-E[Y_{0i}]$
\end{itemize}

\end{frame}



\end{document}